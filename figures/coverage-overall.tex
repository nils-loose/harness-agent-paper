\begin{figure*}[t]
    \centering
    % Row 1: Method-Targeted Coverage (Ours only)
    \textbf{Method-Targeted Coverage}\\[4pt]
    \begin{tabular}{ccccc}
        \includesvg[width=0.18\textwidth]{data/coverage/commons_coverage_comparison.svg} &
        \includesvg[width=0.18\textwidth]{data/coverage/gson_coverage_comparison.svg} &
        \includesvg[width=0.18\textwidth]{data/coverage/guava_coverage_comparison.svg} &
        \includesvg[width=0.18\textwidth]{data/coverage/jackson_coverage_comparison.svg} &
        \includesvg[width=0.18\textwidth]{data/coverage/jsoup_coverage_comparison.svg}
    \end{tabular}

    \vspace{2pt}

    % Row 2: Full Target-Scope Coverage (Ours vs AutoFuzz)
    \textbf{Full Target-Scope Coverage}\\[4pt]
    \begin{tabular}{ccccc}
        \includesvg[width=0.18\textwidth]{data/coverage/package_commons-cli_coverage.svg} &
        \includesvg[width=0.18\textwidth]{data/coverage/package_gson_coverage.svg} &
        \includesvg[width=0.18\textwidth]{data/coverage/package_guava_coverage.svg} &
        \includesvg[width=0.18\textwidth]{data/coverage/package_jackson-databind_coverage.svg} &
        \includesvg[width=0.18\textwidth]{data/coverage/package_jsoup_coverage.svg} \\
        commons-cli & gson & guava & jackson-databind & jsoup
    \end{tabular}

    \caption{Coverage comparison across five Java libraries over 12-hour fuzzing campaigns. \textbf{Top row:} Method-targeted coverage for our generated harnesses, focusing exclusively on target method execution. \textbf{Bottom row:} Full target-scope coverage enabling fair comparison with AutoFuzz baseline. Each plot shows branch coverage percentage over time.}
    \label{fig:coverage-comparison}
\end{figure*}