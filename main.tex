\documentclass[sigconf,review,anonymous]{acmart}

\usepackage{listings}
\usepackage[inkscapeopt={-D}]{svg}
\usepackage{booktabs}
\usepackage{multirow}
\usepackage{pifont}
\usepackage{rotating}
\usepackage{listings}
\usepackage{xcolor}

% YAML syntax highlighting - clean for academic papers
\lstdefinestyle{yaml}{
  basicstyle=\ttfamily\small,
  sensitive=false,
  comment=[l]{\#},
  commentstyle=\color{gray},
  showstringspaces=false,
  emphstyle=\bfseries,
  emph={get_method_code,description,method_signature,type,class_name,default},
  identifierstyle=,
  keywords={null, string},
}

% Java syntax highlighting
\definecolor{javared}{rgb}{0.6,0,0}          % for strings
\definecolor{javagreen}{rgb}{0.25,0.5,0.35}  % for comments
\definecolor{javapurple}{rgb}{0.5,0,0.35}    % for keywords
\definecolor{javadocblue}{rgb}{0.25,0.35,0.75} % for javadoc

\lstset{language=Java,
  basicstyle=\ttfamily\small,
  keywordstyle=\color{javapurple}\bfseries,
  stringstyle=\color{javared},
  commentstyle=\color{javagreen},
  morecomment=[s][\color{javadocblue}]{/**}{*/},
  showstringspaces=false,
  tabsize=2
}

% Define checkmark symbol
\newcommand{\cmark}{\ding{51}}

\setcopyright{acmlicensed}
\copyrightyear{2018}
\acmYear{2018}
\acmDOI{XXXXXXX.XXXXXXX}
\acmConference[Conference acronym 'XX]{Make sure to enter the correct
  conference title from your rights confirmation email}{June 03--05,
  2018}{Woodstock, NY}
\acmISBN{978-1-4503-XXXX-X/2018/06}
\acmSubmissionID{123}
% Define text styles for SVG rendering
\def\axtext{\footnotesize\sffamily}
\def\ticktext{\scriptsize\sffamily}


\begin{document}

\title{Multi-Agent Driven Fuzz-Harness Generation}

%%
%% The "author" command and its associated commands are used to define
%% the authors and their affiliations.
%% Of note is the shared affiliation of the first two authors, and the
%% "authornote" and "authornotemark" commands
%% used to denote shared contribution to the research.
\author{Ben Trovato}
\authornote{Both authors contributed equally to this research.}
\email{trovato@corporation.com}
\orcid{1234-5678-9012}
\author{G.K.M. Tobin}
\authornotemark[1]
\email{webmaster@marysville-ohio.com}
\affiliation{%
  \institution{Institute for Clarity in Documentation}
  \city{Dublin}
  \state{Ohio}
  \country{USA}
}

\author{Lars Th{\o}rv{\"a}ld}
\affiliation{%
  \institution{The Th{\o}rv{\"a}ld Group}
  \city{Hekla}
  \country{Iceland}}
\email{larst@affiliation.org}

\author{Valerie B\'eranger}
\affiliation{%
  \institution{Inria Paris-Rocquencourt}
  \city{Rocquencourt}
  \country{France}
}

\author{Aparna Patel}
\affiliation{%
 \institution{Rajiv Gandhi University}
 \city{Doimukh}
 \state{Arunachal Pradesh}
 \country{India}}

\author{Huifen Chan}
\affiliation{%
  \institution{Tsinghua University}
  \city{Haidian Qu}
  \state{Beijing Shi}
  \country{China}}

\author{Charles Palmer}
\affiliation{%
  \institution{Palmer Research Laboratories}
  \city{San Antonio}
  \state{Texas}
  \country{USA}}
\email{cpalmer@prl.com}

\author{John Smith}
\affiliation{%
  \institution{The Th{\o}rv{\"a}ld Group}
  \city{Hekla}
  \country{Iceland}}
\email{jsmith@affiliation.org}

\author{Julius P. Kumquat}
\affiliation{%
  \institution{The Kumquat Consortium}
  \city{New York}
  \country{USA}}
\email{jpkumquat@consortium.net}

\renewcommand{\shortauthors}{Trovato et al.}


\begin{abstract}
Coverage-guided fuzzing requires high-quality harnesses to exercise library code, yet manual harness creation remains a bottleneck—particularly for Java libraries with complex Maven dependencies, reflection-based initialization, and deep inheritance hierarchies. Existing approaches mine usage patterns from consumer code or derive harnesses from type signatures, but struggle with unavailable corpora or non-obvious API contracts. Recent LLM-based systems face context saturation when preprocessing entire API surfaces and lack mechanisms to interpret coverage feedback semantically.
%
We present a multi-agent architecture that integrates LLM reasoning with targeted code analysis to automate Java harness generation. Five specialized ReAct agents decompose the workflow into research, synthesis, compilation repair, coverage analysis, and refinement—querying documentation, source code, and callgraph information on demand through the Model Context Protocol. Two technical innovations enable effective coverage-guided refinement: method-targeted instrumentation that activates coverage tracking only during target method execution, and agent-guided termination that interprets coverage gaps by examining uncovered methods to distinguish addressable deficiencies from fundamental limitations.
%
Evaluation on seven target methods from six widely-deployed Java libraries (115,000+ Maven dependents) shows our harnesses achieve competitive coverage with manually-written OSS-Fuzz baselines and AutoFuzz at \$3.20 and approximately 10 minutes per harness. During 12-hour fuzzing campaigns, our harnesses triggered 14 crashes revealing 3 unique bugs in production libraries already integrated into OSS-Fuzz.
\end{abstract}
%%
%% The code below is generated by the tool at http://dl.acm.org/ccs.cfm.
%% Please copy and paste the code instead of the example below.
%%
\begin{CCSXML}
<ccs2012>
   <concept>
       <concept_id>10010147.10010257</concept_id>
       <concept_desc>Computing methodologies~Machine learning</concept_desc>
       <concept_significance>500</concept_significance>
       </concept>
   <concept>
       <concept_id>10011007</concept_id>
       <concept_desc>Software and its engineering</concept_desc>
       <concept_significance>500</concept_significance>
       </concept>
   <concept>
       <concept_id>10011007.10011006</concept_id>
       <concept_desc>Software and its engineering~Software notations and tools</concept_desc>
       <concept_significance>500</concept_significance>
       </concept>
   <concept>
       <concept_id>10002978.10003022</concept_id>
       <concept_desc>Security and privacy~Software and application security</concept_desc>
       <concept_significance>500</concept_significance>
       </concept>
 </ccs2012>
\end{CCSXML}

\ccsdesc[500]{Computing methodologies~Machine learning}
\ccsdesc[500]{Software and its engineering}
\ccsdesc[500]{Software and its engineering~Software notations and tools}
\ccsdesc[500]{Security and privacy~Software and application security}

%%
%% Keywords. The author(s) should pick words that accurately describe
%% the work being presented. Separate the keywords with commas.
\keywords{fuzzing, fuzz harness generation, large language models, multi-agent systems, automated software testing, program analysis, feedback-driven refinement, Java ecosystem, coverage-guided fuzzing, software security}


\maketitle

\section{Introduction}

Coverage-guided fuzzing has become the gold standard for discovering vulnerabilities in systems software, yet its effectiveness on library code fundamentally depends on the availability of high-quality fuzz harnesses. A harness bridges unstructured fuzzer inputs to structured API invocations, transforming raw bytes into valid method calls that exercise library logic. Manual harness creation remains the principal bottleneck limiting fuzzing coverage: developers must reverse-engineer API preconditions, synthesize realistic call sequences, and navigate complex dependency graphs. In ecosystems like Java, where libraries require intricate Maven configurations and reflection-based initialization, harness authoring has proven particularly challenging.

Existing automated approaches face complementary limitations, and crucially, most work has focused on C/C++ libraries rather than managed runtime environments. Usage-based methods mine API patterns from client code but require rich consumer corpora—unavailable for niche or newly-released libraries~\cite{DBLP:conf/sigsoft/BabicBCIKKLSW19:FUDGE,DBLP:conf/sp/JeongJYMKJKSH23:UTopia}. Structure-based methods derive harnesses from type signatures but struggle with non-obvious API contracts and often depend on domain-specific heuristics~\cite{DBLP:conf/icse/GreenA22:GraphFuzz,DBLP:conf/icse/ShermanN25:OGHarn}. Feedback-driven refinement shows promise but typically focuses on narrow domains or lacks robust mechanisms to resolve build failures~\cite{DBLP:conf/uss:ZhangLZZZZXLL0H23:Rubick}. While recent LLM-based systems demonstrate progress through coverage-guided prompt evolution~\cite{DBLP:conf/ccs/LyuXCC24:PromptFuzz} and knowledge graph augmentation~\cite{DBLP:conf/icse:XuMZZCHLW25:CKGFuzzer}, Java libraries present distinct challenges: complex Maven dependency resolution, reflection-heavy initialization patterns, and deep inheritance hierarchies that resist simple type-based harness generation. No existing approach provides fully automated, end-to-end synthesis for Java libraries without manual intervention or reliance on external consumer code.

Large language models offer a path forward by combining code synthesis capabilities with learned API usage patterns. However, applying LLMs to harness generation introduces new challenges: preprocessing entire API surfaces exhausts context windows, one-shot generation fails on complex libraries requiring multi-step initialization, and coverage-based refinement risks semantic drift without mechanisms to interpret what coverage gaps actually mean. These challenges demand a fundamentally different approach—one that integrates LLM reasoning with code analysis rather than treating generation as pure synthesis, and that retrieves information on demand rather than preprocessing everything upfront.

We address this through a multi-agent architecture that combines specialized ReAct agents with targeted code analysis. Five agents decompose harness generation into distinct reasoning tasks: research (API exploration through documentation and source code), synthesis (initial code generation), compilation repair (build error diagnosis), coverage analysis (gap interpretation via callgraph and source inspection), and refinement (targeted harness improvement). Rather than preprocessing entire API surfaces into static knowledge graphs, agents query information on demand through the Model Context Protocol—retrieving documentation for specific methods, source code for targeted classes, and callgraph fragments rooted at particular invocations. This query-driven approach maintains focused context while exploring large dependency graphs, enabling the workflow to autonomously handle Maven dependency resolution, reflection-heavy initialization patterns, and iterative compilation repair without manual intervention.

Two technical innovations enable effective coverage-guided refinement. First, we introduce method-targeted coverage instrumentation through runtime-toggled JaCoCo tracking that activates only during target method execution. Standard instrumentation measures all executed code, creating misaligned incentives where harnesses invoke unrelated utility methods to inflate metrics; our scoping ensures coverage reflects target behavior rather than incidental framework initialization. Second, we delegate termination decisions to an agent that interprets coverage gaps by examining uncovered methods' source code to distinguish addressable deficiencies (missing input variants, unexplored API paths) from fundamental limitations (unreachable defensive code, external I/O dependencies). This interpretation enables stopping when refinement yields diminishing returns while continuing when concrete improvement strategies exist.

We evaluate our approach on seven target methods from six widely-deployed Java libraries (commons-cli, gson, guava, jackson-databind, jsoup, antlr4) totaling 115,000+ Maven dependents. Our generated harnesses match or outperform coverage achieved by OSS-Fuzz's manually-written harnesses and Jazzer's AutoFuzz reflection-based generation. Generation costs average \$3.20 and approximately 10 minutes per harness, making the approach practical for iterative fuzzing workflows. Critically, our harnesses triggered 14 crashes during 12-hour fuzzing campaigns, revealing 3 unique bugs (2 in commons-cli, 1 in jsoup)—demonstrating that automatically-generated harnesses achieve sufficient input diversity to find real vulnerabilities in production libraries already integrated into OSS-Fuzz. In one striking example (ANTLR4), our generated harness achieved measurable coverage on a method where the existing OSS-Fuzz harness registered 0\% coverage throughout the campaign, unable to satisfy preconditions required to reach the target.

This work makes the following contributions:
\begin{itemize}
    \item A multi-agent architecture integrating LLM reasoning with code analysis to fully automate harness generation for Java libraries.

    \item A query-driven tool interface using the Model Context Protocol that retrieves precisely scoped information on demand, preventing the context saturation that limits one-shot generation.

    \item Method-targeted coverage instrumentation with agent-guided termination that interprets coverage gaps through source code analysis rather than applying fixed thresholds.

    \item Empirical validation demonstrating competitive coverage with OSS-Fuzz baselines, practical generation costs (\$3.20, approximately 10 minutes per harness), and discovery of 3 unique bugs in production libraries.
\end{itemize}


\begin{figure*}[t]
    \centering
    \includesvg[width=.9\linewidth]{data/overview.svg}
    \caption{Schematic overview of the harness generation workflow.}
    \label{fig:overview}
\end{figure*}

\section{Preliminaries}

This section establishes the technical foundation for our approach, introducing coverage-guided fuzzing for managed runtimes, the role of large language models in code synthesis, and relevant baseline systems.

\subsection{Coverage-Guided Fuzzing and Harness Design}
\label{sec:prelim:fuzzing}

Coverage-guided fuzzing operates by repeatedly executing a program with generated inputs while monitoring code coverage feedback. Each execution provides feedback, typically in the form of newly covered branches or basic blocks, that guides subsequent input mutations toward unexplored program regions. This feedback loop enables fuzzers to systematically navigate complex input spaces and discover deep bugs that evade random testing.

Applying coverage-guided fuzzing to library code introduces a fundamental challenge: libraries expose APIs rather than standalone executables, requiring a \emph{fuzz harness} to mediate between the fuzzer and the target. A harness transforms unstructured byte streams from the fuzzer into valid API invocations by: (1)~parsing fuzz input into structured data types, (2)~constructing necessary object state and satisfying initialization preconditions, (3)~invoking target methods with derived arguments, and (4)~handling or observing exceptions and abnormal termination. The quality of this translation directly determines fuzzing effectiveness: shallow harnesses that merely parse inputs yield limited coverage, while well-designed harnesses that exercise complex API interactions expose deeper program logic and uncover latent bugs.

\subsection{Jazzer: Coverage-Guided Fuzzing for the JVM}
\label{sec:prelim:jazzer}

Jazzer~\cite{CITE:Jazzer} brings coverage-guided fuzzing to the Java Virtual Machine through a hybrid architecture combining native code instrumentation with JVM bytecode manipulation. The fuzzer executes as a native process that invokes Java methods via JNI, while a Java agent instruments target bytecode at class-load time to collect coverage feedback. This design enables efficient in-process fuzzing with fine-grained branch coverage, comparable to AFL and libFuzzer for native code.

Jazzer harnesses implement a standard interface that provides methods for consuming integers, strings, booleans, and other primitive types. Harness authors use this interface to construct valid inputs and orchestrate API calls. While manual harness creation remains common practice, Jazzer includes an AutoFuzz mode that leverages Java reflection to automatically generate harnesses. AutoFuzz discovers accessible constructors and methods, recursively building required objects and mapping fuzzer bytes to parameter types. 


\subsection{Large Language Models for Code Synthesis}
\label{sec:prelim:llm}

Large Language Models (LLMs) are transformer-based neural architectures trained on massive corpora of source code and natural language to predict and generate token sequences. Through next-token prediction over billions of parameters, LLMs internalize syntactic patterns, API usage idioms, and semantic relationships between code elements. When conditioned on appropriate context—such as function signatures, documentation, or partial implementations—LLMs can synthesize plausible code completions, generate test cases, or repair buggy programs~\cite{CITE:Codex,CITE:AlphaCode}.

Recent work has demonstrated that LLMs can be organized into multi-agent systems where distinct instances specialize in complementary subtasks~\cite{CITE:MultiAgentCode}. Rather than relying on a single monolithic prompt, multi-agent architectures decompose complex objectives into stages handled by specialized agents that communicate through structured message passing or shared state. For code generation tasks, this decomposition enables separation of concerns: one agent explores API documentation and infers usage patterns, another synthesizes implementations, while a third diagnoses compilation errors and proposes fixes. This division of labor mirrors human collaborative workflows and has been shown to improve both generation quality and success rates on challenging benchmarks~\cite{CITE:AgentBench}.

\subsection{Tool-Augmented LLM Reasoning}
\label{sec:prelim:tools}

While LLMs demonstrate impressive code understanding from pre-training, their knowledge is static and limited to patterns observed in training data. \emph{Tool-augmented reasoning} addresses this limitation by equipping LLMs with external capabilities they can invoke during generation~\cite{DBLP:conf/nips/SchickDSHWSCSW23:Toolformer,CITE:ToolLLM}. For code synthesis tasks, relevant tools include compilers (to validate syntax), test executors (to verify functional correctness), static analyzers (to extract API signatures), and documentation retrievers (to ground generation in current libraries).

The ReAct (Reasoning and Acting) paradigm~\cite{DBLP:conf/iclr/YaoZYDN023:ReAct} formalizes tool use as an interleaved process: the model alternates between reasoning steps (generating natural language explanations of its strategy) and action steps (invoking tools and observing outputs). This loop continues until the model determines it has sufficient information to complete the task. ReAct-style agents have demonstrated substantial improvements over direct prompting on tasks requiring information retrieval, calculation, or interaction with external systems.

\subsection{Baseline: OSS-Fuzz-Gen}
\label{sec:prelim:ossfuzzgen}

OSS-Fuzz-Gen~\cite{CITE:OSSFuzzGen} represents the current state-of-the-art in LLM-driven harness generation. It combines lightweight static analysis with iterative LLM prompting to synthesize harnesses for libraries integrated into Google's OSS-Fuzz continuous fuzzing service. The system operates in three phases: (1)~static analysis extracts function signatures and type information, (2)~an LLM generates candidate harnesses conditioned on templates encoding language-specific patterns (e.g., Java exception handling, C++ RAII), and (3)~compilation and short fuzzing trials filter invalid or low-coverage candidates.
OSS-Fuzz-Gen's Java extension provides language-specific prompt templates that guide harness structure for Java libraries. 


\section{Automated Harness Generation}%
\label{sec:approach}
We present an agent-based approach to automated fuzzing harness generation that addresses several fundamental challenges of generating effective harnesses for complex library APIs. Our approach combines LLM-powered agents with static analysis and dynamic coverage feedback to iteratively construct and refine harnesses that achieve deep code coverage.
%
\subsection{Core Principles}%
\label{subsec:core-principles}
Our approach is guided by three principles that inform the design of our agent-based harness generation system.
\paragraph{Iterative Contextual Exploration}
API semantics vary widely across libraries. Initialization sequences, factory patterns, and implicit preconditions resist capture in fixed schemas~\cite{CITE:template-based-test-gen}. Rather than attempting one-shot extraction~\cite{CITE:one-shot-LLM-generation}, we employ agents that maintain full reasoning history and iterate freely, following information dependencies as they emerge. An agent examining a method signature may discover complex parameter types, prompting investigation of factory methods. This iterative exploration accommodates API diversity without exhaustive preprocessing. We instantiate this through ReAct agents~\cite{yao2023react} with modular decomposition preventing context saturation while preserving reasoning continuity.

\paragraph{Query-Driven Information Access}
The information space surrounding any target method (reachable code, transitive dependencies, documentation) vastly exceeds LLM context windows. Rather than exhaustively preprocessing~\cite{CITE:static-analysis-preprocessing} or using retrieval heuristics~\cite{CITE:RAG-for-code}, we provide agents with query tools that return precisely requested information. Agents construct context based on reasoning needs, maintaining high signal-to-noise ratios. Role-based tool restrictions ensure agents focus on task-pertinent information, preventing exploration drift.

\paragraph{Agent-Guided Refinement Termination}
Coverage metrics are fundamentally ambiguous. Incomplete coverage may indicate inadequate harnesses, unreachable code, or paths requiring external resources. Rather than applying fixed thresholds~\cite{CITE:coverage-threshold-approaches}, we delegate termination decisions to an agent that interprets coverage gaps by examining uncovered code (control flow, preconditions, reachability) to determine whether gaps are addressable. This interpretation enables early stopping when refinement yields diminishing returns while continuing when concrete improvement strategies exist, avoiding both premature termination and wasteful iteration.

\subsection{Workflow Architecture}%
\label{subsec:workflow-architecture}


Figure~\ref{fig:overview} illustrates our workflow as a sequence of transformations that progressively refine a fuzzing harness. After initializing the environment by downloading library artifacts and preparing analysis infrastructure, the workflow proceeds through three phases. Each phase transforms intermediate artifacts through agent-driven reasoning and automated validation.

\paragraph{Target Research}
The workflow begins by transforming the target method signature into contextual knowledge about API semantics. A research agent queries documentation and examines source code to understand how the target method should be invoked. Rather than exhaustively extracting all available information, the agent follows its reasoning to identify relevant patterns such as required initialization sequences, factory method usage, or implicit preconditions. The agent produces a natural language research report structured with predefined markdown sections that organize findings without constraining content to rigid schemas, accommodating diverse API designs.

\paragraph{Harness Construction}
The research report is transformed into executable code through two sequential steps. First, a generation agent synthesizes initial harness code that instantiates the target method with fuzzer-generated inputs. Second, a compilation agent iteratively resolves build errors by analyzing compiler diagnostics and querying additional source code to correct syntactic issues. Once compilation succeeds, the workflow transitions to the coverage analysis phase.

\paragraph{Coverage Analysis and Refinement}
The compiled harness is executed under fuzzing to collect coverage data. Coverage instrumentation tracks only code executed while the target method is active, ensuring metrics reflect the target's behavior rather than incidental framework initialization. Coverage data merged with static callgraph information becomes the input to an iterative refinement loop. A coverage analysis agent examines which reachable methods remain uncovered, explores their semantics through source code and documentation queries, and determines whether coverage gaps are addressable through harness modifications. If the agent identifies concrete improvement opportunities, a refinement agent modifies the harness to exercise uncovered paths and the workflow returns to compilation and fuzzing. This loop continues until the coverage agent determines refinement yields diminishing returns or an iteration limit is reached.
\subsection{Static Analysis and Instrumentation}%
\label{subsec:static-analysis}
\begin{lstlisting}[float=tbh,language=Java,caption={Harness showing selective coverage instrumentation through runtime control of JaCoCo's recording state. The coverage handling calls are added using offline instrumentation after compilation.},label={lst:coverage-tracking},float=htbp,basicstyle=\small\ttfamily,numbers=left,numberstyle=\tiny\color{gray},frame=single]
public static void fuzzerTestOneInput(
      FuzzedDataProvider data) {
   RT.getAgent().setRecording(false); 
   // Parameter/ Instance preparation
   Options o = prepareOptions(data);
   try{
      RT.getAgent().setRecording(true);
      Parser.parse(o);
      RT.getAgent().setRecording(false);
   } catch (IllegalArgumentException var15) {
        // Expected exception 
      }
}
\end{lstlisting}
Our approach requires three preprocessing artifacts that enable efficient agent exploration and accurate coverage measurement: parsed API documentation, static callgraphs, and instrumented coverage collection.

\paragraph{API Documentation Indexing}
We extract method signatures, parameter types, and semantic descriptions from Javadoc HTML archives distributed with Maven artifacts, parsing them using Beautiful Soup~\cite{beautifulsoup}. This provides agents with concise API contracts, avoiding context pollution from verbose source code when only interface information is needed. Parsed documentation is indexed for tool usage.

\paragraph{Static Callgraph Construction}
We compute a static callgraph rooted at the target method using SootUp's Class Hierarchy Analysis~\cite{sootup2023}, traversing method invocations to a configurable depth. We use depth~10 by default, reducing to depth~5 for large libraries where deeper analysis becomes computationally prohibitive. Each node records the method's signature, enclosing class, and distance from the target. This callgraph serves two purposes: it scopes coverage analysis to reachable methods, and it provides agents with structural context about the target's call dependencies. For libraries with complex inheritance or external dependencies, we include transitive dependencies in the analysis to ensure accurate call resolution.

\paragraph{Method-Targeted Coverage}
Standard coverage instrumentation measures all executed code, creating a misaligned incentive for agents to invoke unrelated utility methods. We address this by extending JaCoCo~\cite{jacoco} with lightweight runtime toggling of coverage tracking that is accessible through the runtime API. After successful compilation of the target harness we use ASM~\cite{ASM} to add the required call wrapping using lightweight offline instrumentation. As shown in Listing~\ref{lst:coverage-tracking}, the harness disables coverage recording during initialization (line~3), enables it immediately before invoking the target method (line~9), and disables it again upon return (line~11). This scoping ensures coverage metrics reflect the target's behavior rather than incidental framework code, aligning agent optimization with the goal of exploring the target method's logic.
%
%
\subsection{Tool-Augmented Exploration}%
\label{subsec:tool-augmented-exploration}
\begin{figure}[t]
    \centering
    \includesvg[width=.9\linewidth]{data/mcp-design.svg}
    \caption{Overview of exposed tools through the model context protocol (MCP).}
    \label{fig:mcp-design}
\end{figure}

The preprocessing artifacts described in Section~\ref{subsec:static-analysis} must be exposed to agents in a form compatible with LLM tool-use protocols. While we provide initial context where immediately relevant (e.g., target method documentation at agent initialization, as detailed in subsequent workflow sections), exhaustively injecting all documentation, source code, and callgraph data would exhaust token budgets. Instead, we implement a query-based interface using the Model Context Protocol (MCP)~\cite{mcp2024}, enabling agents to retrieve additional information on demand as reasoning progresses.


\paragraph{Tool Interface Design}
We expose three tool categories corresponding to the preprocessing artifacts: (1)~\emph{documentation queries} retrieve package, class or method signatures and parameter descriptions from indexed Javadoc, (2)~\emph{source code retrieval} returns method implementations for examining initialization logic or exception handling, and (3)~\emph{callgraph queries} provide reachability information rooted at the target method. Figure~\ref{fig:mcp-design} illustrates the initialization process and interaction with underlying tooling. Each tool accepts structured parameters (e.g., class name, method signature) and returns responses optimized for LLM consumption: concise method signatures, minimal code snippets, and depth-limited callgraph fragments. Figure~\ref{fig:mcp-tool-schema} shows the schema for a source code retrieval tool, demonstrating how parameter descriptions guide agents toward correct tool usage.
\begin{table}[t]
\caption{Tool availability across DRAG workflow agents. MCP tools are provided by three Model Context Protocol servers that enable LLM agents to query code intelligence APIs.}
\centering
\setlength{\tabcolsep}{4pt}
\renewcommand{\arraystretch}{1.1}
\begin{tabular}{c l | c c c c c}
\toprule
& \multirow{2}{*}{\textbf{Tool}} & \multicolumn{5}{c}{\textbf{ReAct Agents}} \\
\cmidrule(lr){3-7}
& & \textbf{RSH} & \textbf{GEN} & \textbf{PAT} & \textbf{CVA} & \textbf{REF} \\
\midrule
\multirow{6}{*}{\rotatebox{90}{\textit{Docs}}}
& \texttt{method\_doc} & \cmark & \cmark & \cmark & \cmark & \cmark \\
& \texttt{class\_doc} & \cmark & \cmark & \cmark & \cmark & \cmark \\
& \texttt{package\_doc} & \cmark & \cmark & \cmark & \cmark & \cmark \\
& \texttt{list\_packages} & \cmark & \cmark & \cmark & \cmark & \cmark \\
& \texttt{list\_classes} & \cmark & \cmark & \cmark & \cmark & \cmark \\
& \texttt{list\_methods} & \cmark & \cmark & \cmark & \cmark & \cmark \\
\midrule
\multirow{5}{*}{\rotatebox{90}{\textit{Code}}}
& \texttt{get\_code} & \cmark & \cmark & \cmark & \cmark & \cmark \\
& \texttt{find\_def} & \cmark & \cmark & \cmark & \cmark & \cmark \\
& \texttt{find\_refs} & \cmark & \cmark & \cmark & \cmark & \cmark \\
& \texttt{grep} & \cmark & \cmark & \cmark & \cmark & \cmark \\
& \texttt{find\_symbol} & \cmark & \cmark & \cmark & \cmark & \cmark \\
\midrule
\multirow{2}{*}{\rotatebox{90}{\textit{CG}}}
& \texttt{reach\_methods} & -- & -- & -- & (\cmark) & -- \\
& \texttt{path\_to\_method} & -- & -- & -- & \cmark & -- \\
\midrule
\multirow{2}{*}{\rotatebox{90}{\textit{Exec}}}
& \texttt{compiler} & -- & -- & (\cmark) & -- & -- \\
& \texttt{jazzer} & -- & -- & -- & (\cmark) & -- \\
\bottomrule
\end{tabular}
\smallskip
\begin{flushleft}
\footnotesize
\textbf{Agents:} RSH (Research), GEN (Generation), PAT (Patching), CVA (Coverage Analysis), REF (Refinement).
\cmark~= Available, (\cmark)~= Static invocation (not queryable), --~= No access.
\textbf{Categories:} Docs (Javadoc API documentation); Code (source code indexing); Graph (call graph analysis); Exec (compiler and fuzzer).
\end{flushleft}
\vspace{-3mm}
\label{tab:tool-availability}
\end{table}

\paragraph{Role-Based Tool Access}
To prevent exploration drift, we restrict tool access based on agent role. Table~\ref{tab:tool-availability} summarizes tool availability across the five ReAct agents in our workflow. All agents access documentation (\textit{Docs}) and source code (\textit{Code}) tools, which provide foundational API understanding throughout the workflow. However, callgraph tools (\textit{CG}) are restricted to the coverage analysis agent, which uses reachability information to interpret coverage gaps during refinement. Execution tools (\textit{Exec}) are invoked statically through state transitions rather than through MCP queries. This access control maintains focus on phase-specific objectives, reducing wasted tool invocations and preventing agents from pursuing information irrelevant to their current task.

\begin{figure}[t]
\begin{lstlisting}[style=yaml,caption={MCP tool schema for the \texttt{get\_method\_code} tool from CodeContextMCP server. Tools are exposed to ReAct agents based on tag-based access control.},label={fig:mcp-tool-schema},numbers=left,numberstyle=\tiny\color{gray},frame=single]
get_method_code:
  description:
    Get source code for a specific method
  method_signature:
    type: string
    description:
      Source code method signature (e.g.,
      'String someMethod(String, int)' or
      'someMethod(String, int)'). If possible 
      provide full signature including parameters 
      for precise matching, however to get all 
      overloads 'someMethod' also works.
  class_name:
    type: string
    description:
      (Fully qualified) class name (e.g.,
      'com.example.Util'). If possible, use the
      fully qualified name. Required for accurate
      matching!
    default: null
\end{lstlisting}
\end{figure}
%
\subsection{Target Research}%
\label{subsec:target-research}

Following environment initialization (Maven download, documentation extraction, callgraph construction), the research agent transforms the target method signature into contextual knowledge about API semantics. The agent is initialized with the target method's signature, documentation, and source code, then iteratively queries additional documentation and source code through MCP tools. Figure~\ref{fig:research-sequence} illustrates this query-driven exploration pattern. Rather than exhaustively extracting all available information, the agent follows its reasoning to identify relevant patterns: required initialization sequences, factory method usage, and implicit preconditions. The agent produces a natural language research report structured with predefined markdown sections that organize findings without constraining content to rigid schemas, accommodating diverse API designs.
\begin{figure}[thb]
\centering
\includesvg[width=.9\linewidth]{data/tool-usage-sequence.svg}
\caption{Sequence diagram showing initial tool interactions during the research phase for \texttt{Jsoup.parse(String)}. Solid arrows represent tool calls, dashed arrows represent responses.}
\label{fig:research-sequence}
\end{figure}
\subsection{Harness Generation}%
\label{subsec:harness-generation}

The research report is transformed into compilable code through two sequential steps: generation and compilation. The generation agent synthesizes initial harness code that instantiates the target method with fuzzer-generated inputs. The agent has access to the Jazzer API documentation and queries additional source code to resolve ambiguities in constructor signatures or factory method usage. A critical aspect of harness synthesis is exception handling: the agent must determine which exceptions represent expected API behavior (e.g., IllegalArgumentException for invalid inputs) that should be caught to continue fuzzing, versus unexpected exceptions that indicate bugs and must propagate to Jazzer's crash detection. The agent analyzes API documentation and method signatures to infer expected exception contracts, synthesizing appropriate try-catch blocks that preserve bug-finding capability. The agent outputs harness source code and a list of Maven dependencies. Separating research from generation prevents context saturation: research explores broadly without committing to code structure, while generation focuses narrowly on producing syntactically valid harness code.

If the compile step fails, a compilation agent iteratively resolves build errors by analyzing compiler diagnostics, querying source code and documentation to understand the root cause, and producing corrected code until compilation succeeds or an iteration limit is reached. Common error patterns include missing imports, incorrect method signatures, and improper exception handling. Independent compilation repair allows targeted iteration budgets distinct from initial synthesis, reflecting the need for precise syntactic correctness in the final artifact.
\subsection{Coverage-Guided Refinement}%
\label{subsec:coverage-guided-refinement}

Once compilation succeeds, the compiled harness is instrumented and executed under fuzzing to collect initial coverage data. Coverage instrumentation tracks only code executed while the target method is active, ensuring metrics reflect the target's behavior rather than incidental framework initialization. An iterative refinement loop then uses this coverage feedback to improve harness effectiveness through two collaborative agents: a coverage analysis agent that interprets coverage gaps and decides whether refinement is worthwhile, and a refinement agent that modifies the harness.

\paragraph{Coverage Analysis and Termination}
To seed the coverage analysis, we merge method-level coverage data with the static callgraph to produce an annotated view showing coverage status for each reachable method, grouped by call depth from the target. The coverage analysis agent explores uncovered or partially covered methods by querying their source code and documentation to determine whether gaps reflect addressable harness deficiencies (missing input diversity, unexplored API paths) or fundamental limitations (unreachable defensive code, external I/O dependencies). The agent then makes a termination decision: stop if further refinement yields diminishing returns, or continue with a strategy targeting specific uncovered methods.

\paragraph{Harness Refinement}
If refinement continues, the refinement agent receives the current harness code, the coverage analysis strategy (priority methods and improvement rationale), and annotated coverage data. The agent modifies the harness to exercise uncovered code paths through strategies such as diversifying input generation, invoking alternative API paths, or triggering exception handlers through edge-case inputs. The refined harness re-enters compilation and fuzzing, creating a feedback loop that continues until the coverage agent determines refinement yields diminishing returns or an iteration limit is reached. Convergence detection through code hashing prevents oscillation between semantically equivalent harness variants.
\subsection{Implementation}%
\label{subsec:implementation}
We implement our approach using LangGraph~\cite{langgraph2024} for workflow orchestration and Claude 4.5 Sonnet as the underlying model. The workflow state machine coordinates agent execution, managing transitions between research, generation, compilation, fuzzing, and refinement phases. Agents interact with preprocessed artifacts (documentation, source code, callgraphs) through a standardized tool protocol, with tool responses cached to avoid redundant queries. Harnesses are compiled using gralde and executed using Jazzer with instrumented coverage collection.

\section{Evaluation}%
\label{sec:evaluation}

We evaluate our harness generation approach on widely used Maven libraries, examining achieved coverage, computational costs, and agent behavior patterns. Our evaluation demonstrates that the approach produces competitive harnesses to existing baselines and techniques while maintaining practical generation costs. Additionally, during the 12-hour fuzzing campaigns, our harnesses uncovered multiple previously unknown bugs in mature libraries, validating their effectiveness in real-world testing scenarios.

\subsection{Experimental Setup}%
\label{subsec:exp-setup}
\begin{table*}[t]
\caption{Benchmark library characteristics. All libraries are widely-deployed in the Maven ecosystem and represent real-world fuzzing targets.}
\centering
\setlength{\tabcolsep}{3pt}
\renewcommand{\arraystretch}{1.1}
\small
\begin{tabular}{l r r l l l r}
\toprule
\textbf{Library} & \textbf{Version} & \textbf{Dependents} & \textbf{Class Name} & \textbf{Target Method} & \textbf{Category} & \textbf{Rank} \\
\midrule
commons-cli      & 1.10.0     & 5K+  & DefaultParser        & parse(Options, String[])              & CLI Parser       & \#1 \\
gson             & 2.13.1     & 27K+ & JsonParser           & parseString(String)                   & JSON Library     & \#2 \\
guava            & 33.4.8-jre & 42K+ & HostAndPort          & fromString(String)                    & Core Utilities   & \#1 \\
jackson-databind & 2.20.0     & 36K+ & ObjectMapper         & readTree(String)                      & JSON Library     & \#1 \\
jsoup            & 1.21.1     & 4K+  & Jsoup                & parse(String)                         & HTML Parser      & \#1 \\
\midrule
\multirow{2}{*}{antlr4} & \multirow{2}{*}{4.13.2} & \multirow{2}{*}{1K+} & Grammar & Grammar(String) & \multirow{2}{*}{Parser Generator} & \multirow{2}{*}{\#1} \\[-.9pt]
                &            &      & Grammar              & createParserInterpreter(TokenStream)  &                  &      \\
\bottomrule
\end{tabular}
\smallskip
\begin{flushleft}
\footnotesize
Dependents = Maven artifacts declaring this library as a dependency. Rank = Category ranking on Maven Central.
\end{flushleft}
\vspace{-3mm}
\label{tab:benchmarks}
\end{table*}

We evaluate on seven target methods from six widely-deployed Java libraries (Table~\ref{tab:benchmarks}). The selected targets span parsers (commons-cli, jsoup, antlr4), JSON libraries (gson, jackson-databind), and core utilities (guava).
All selected target methods have existing harnesses in OSS-Fuzz~\cite{CITE:OSSFuzz}, which serve as our primary baseline. We additionally compare against Jazzer AutoFuzz~\cite{CITE:Jazzer}, which uses Java reflection for automatic harness generation. For LLM-based comparison, we attempted to use OSS-Fuzz-Gen~\cite{CITE:OSSFuzzGen}, but encountered implementation issues, primarily in model output parsing, that prevented successful harness generation for the selected targets Java targets.
\par
To measure the effectiveness of fuzzing the targeted method, we measure coverage under two configurations: (1)~method-targeted coverage activates only during target method execution (Section~\ref{subsec:static-analysis}), focusing metrics on target behavior; (2)~full target-scope coverage uses standard JaCoCo instrumentation across the entire library for fair baseline comparison.
All campaigns are run for 12 hours per target with a single fuzzing thread and an empty seed corpus. 
\subsection{Coverage Effectiveness}%
\label{subsec:coverage-effectiveness}

\begin{figure*}[t]
    \centering
    % Row 1: Method-Targeted Coverage (Ours only)
    \textbf{Method-Targeted Coverage}\\[4pt]
    \begin{tabular}{ccccc}
        \includesvg[width=0.18\textwidth]{data/coverage/commons_coverage_comparison.svg} &
        \includesvg[width=0.18\textwidth]{data/coverage/gson_coverage_comparison.svg} &
        \includesvg[width=0.18\textwidth]{data/coverage/guava_coverage_comparison.svg} &
        \includesvg[width=0.18\textwidth]{data/coverage/jackson_coverage_comparison.svg} &
        \includesvg[width=0.18\textwidth]{data/coverage/jsoup_coverage_comparison.svg}
    \end{tabular}

    \vspace{2pt}

    % Row 2: Full Target-Scope Coverage (Ours vs AutoFuzz)
    \textbf{Full Target-Scope Coverage}\\[4pt]
    \begin{tabular}{ccccc}
        \includesvg[width=0.18\textwidth]{data/coverage/package_commons-cli_coverage.svg} &
        \includesvg[width=0.18\textwidth]{data/coverage/package_gson_coverage.svg} &
        \includesvg[width=0.18\textwidth]{data/coverage/package_guava_coverage.svg} &
        \includesvg[width=0.18\textwidth]{data/coverage/package_jackson-databind_coverage.svg} &
        \includesvg[width=0.18\textwidth]{data/coverage/package_jsoup_coverage.svg} \\
        commons-cli & gson & guava & jackson-databind & jsoup
    \end{tabular}

    \caption{Coverage comparison across five Java libraries over 12-hour fuzzing campaigns. \textbf{Top row:} Method-targeted coverage for our generated harnesses, focusing exclusively on target method execution. \textbf{Bottom row:} Full target-scope coverage enabling fair comparison with AutoFuzz baseline. Each plot shows branch coverage percentage over time.}
    \label{fig:coverage-comparison}
\end{figure*}

Figure~\ref{fig:coverage-comparison} shows line coverage over time across five Java libraries. The top row compares method-targeted coverage for our generated harnesses against the OSS-Fuzz baseline, demonstrating focused execution of target method logic. The bottom row shows full target-scope coverage comparing our approach against AutoFuzz.
% TODO After plots are finished
The generated harnesses match or outperform coverage across all evaluated targets. Under full target-scope instrumentation, our harnesses reach [TODO: X\%] average coverage compared to AutoFuzz's [TODO: Y\%], representing a [TODO: Z\%] relative improvement.
%
%\paragraph{Coverage Trajectory Analysis}
%Beyond final coverage values, the temporal dynamics reveal important differences in how harnesses explore target code. Our harnesses typically reach 80\% of their final coverage within the first hour, indicating effective initial synthesis that exercises core API paths immediately. In contrast, AutoFuzz exhibits slower convergence, often requiring 2-3 hours to plateau. This suggests that reflection-based type instantiation discovers shallow API interactions quickly but struggles to synthesize complex call sequences required for deeper coverage.
%
%Notably, gson and jackson-databind show continued coverage growth throughout the 12-hour campaign for our harnesses, while AutoFuzz plateaus within 3-4 hours. This sustained exploration indicates that our harnesses exercise diverse input patterns that trigger different code paths, whereas AutoFuzz's deterministic type coercion limits input space diversity.
%
\par
\begin{figure}[t]
    \centering
    \begin{tabular}{cc}
        \includesvg[width=0.45\columnwidth]{data/coverage/antlr4-4.13.2_edited_coverage_comparison.svg} &
        \includesvg[width=0.45\columnwidth]{data/coverage/antlr4-4.13.2_normal_coverage_comparison.svg} \\
        (a) antlr4 (edited) & (b) antlr4 (normal)
    \end{tabular}
    \caption{Coverage comparison for ANTLR4 library with edited and normal harness variants.}
    \label{fig:antlr4-coverage}
\end{figure}

To demonstrate the effect of targeted harness generation we evaluate the existing OSS-Fuzz harness for ANTLR4~\footnote{https://github.com/google/oss-fuzz/blob/master/projects/antlr4-java/GrammarFuzzer.java} with targeted method coverage. The OSS-Fuzz harness exercises multiple methods in a single driver, including grammar and subsequently parser interpreter creation. Notably, the effective tesiting of the parser interpreter relies on the prior successful creation of a valid grammar object. To evlaute the respective coverage we modify the harness once by only creating the grammar object (Figure~\ref{fig:antlr4-coverage}(a)) and once by using the harness as-is (Figure~\ref{fig:antlr4-coverage}(b)). The coverage for each harness is then measured with method-targeted coverage for the respective target method. Additionally, we compare against two of our automatically generated harnesses that each target one of the two methods individually. The results show that both of our generated harnesses outperform the existing harness in both scenarios. For the unmodified harness it is apparent that it does not even reach the target method createParserInterpreter as the coverage remains at 0\% throughout the fuzzing campaign whereas our generated handness succesfully exersised this method. While both these campains were run with identical confugrations as described above, consistently after 30 minutes jazzer runs a test an artifact that triggers a timeout causing jazzer to terminate. We were unable to configure jazzer to continue running. Yet, the general trend is observable.
%
\subsection{Bug Discovery}%
\label{subsec:bug-discovery}
Beyond coverage metrics, we examine whether our generated harnesses discover actual bugs during fuzzing campaigns. Jazzer reports crashes through its exception handling infrastructure, distinguishing between genuine crashes (uncaught exceptions indicating bugs) and caught exceptions that represent normal control flow. This presents are core challenge for the harness generation, as the generated harnesses must avoid over-catching exceptions that would mask real bugs while also prevent spurious crashes from expected error handling.
Across the 12-hour fuzzing campaigns, the generated harnesses triggered a total of 14 crashes in two libraries. After manual investigation we determine, that all reported crashes represent genuine bugs in the target library. The harnesses correctly identified uncaught exceptions rather than reporting false positives from expected exception handling. Manual triage revealed 3 unique bugs:

\begin{itemize}
\item \textbf{commons-cli}: Two distinct null pointer exceptions in option parsing logic, triggered by edge-case combinations of option configurations and malformed arguments. The crashes manifest in both long-option and short-option code paths, with 12 total crash artifacts reducing to 2 unique root causes.
\item \textbf{jsoup}: One index-out-of-bounds exception in HTML tree building logic, triggered by complex malformed HTML input (~1KB). This crash represents a potential denial-of-service vector, as attackers could craft HTML to crash the parser. Two crash artifacts correspond to the same underlying bug.
\end{itemize}

We are currently coordinating responsible disclosure with library maintainers. The commons-cli bugs represent robustness issues, while the jsoup bug has higher severity due to its potential impact on server-side HTML processing.
These results demonstrate that our automatically-generated harnesses achieve sufficient input diversity and API coverage to discover real bugs in mature, widely-deployed libraries that already have existing harnesses in the OSS-Fuzz ecosystem. The fact that Jazzer reported zero false positives highlights the effectiveness of the harness generation in balancing exception handling to expose genuine bugs without over-catching.
%
\subsection{Generation Costs and Agent Behavior}%
\label{subsec:generation-costs}
%
\begin{table}[htb]
\caption{Harness generation cost and agent activity across target libraries. Agent rows show iterations and tool calls, with multiple values (e.g., 5/10) indicating successive refinement rounds. ANTLR4 targets discussed separately in Section~\ref{subsec:coverage-effectiveness}; averages include all seven target methods.}
\centering
\setlength{\tabcolsep}{3pt}
\renewcommand{\arraystretch}{1.1}
\small
\begin{tabular}{l l | r r r r r | r}
\toprule
& \textbf{Metric} & \textbf{commons-cli} & \textbf{gson} & \textbf{guava} & \textbf{jackson} & \textbf{jsoup} & \textbf{Avg} \\
\midrule
\multirow{3}{*}{\rotatebox{90}{\textbf{Total}}} & Tokens & 1896K & 557K & 395K & 1285K & 1039K & 982K \\
& Cost & \$6.25 & \$1.81 & \$1.34 & \$4.13 & \$3.32 & \$3.20 \\
& Time & 1200s & 337s & 417s & 684s & 551s & 599s \\
\midrule
\multirow{2}{*}{\rotatebox{90}{\textbf{RES}}} & Iter & 17 & 17 & 16 & 17 & 19 & 17.7 \\
& Tools & 44 & 27 & 23 & 24 & 28 & 30.7 \\
\midrule
\multirow{2}{*}{\rotatebox{90}{\textbf{GEN}}} & Iter & 3 & 2 & 3 & 3 & 4 & 3.1 \\
& Tools & 5 & 2 & 2 & 3 & 4 & 3.4 \\
\midrule
\multirow{2}{*}{\rotatebox{90}{\textbf{PAT}}} & Iter & 5/0/0/0 & 0 & 0/0 & 6/0 & 0/0 & 0.9 \\
& Tools & 11/0/0/0 & 0 & 0/0 & 6/0 & 0/0 & 1.2 \\
\midrule
\multirow{2}{*}{\rotatebox{90}{\textbf{CVA}}} & Iter & 16/14/13/14 & 17 & 4/4 & 18/14 & 14/18 & 12.5 \\
& Tools & 28/22/16/28 & 31 & 4/5 & 21/16 & 24/21 & 18.3 \\
\midrule
\multirow{2}{*}{\rotatebox{90}{\textbf{REF}}} & Iter & 10/7/15 & -- & 5 & 8 & 10 & 8.9 \\
& Tools & 19/15/23 & -- & 5 & 13 & 22 & 16.1 \\
\bottomrule
\end{tabular}
\smallskip
\begin{flushleft}
\footnotesize
\textbf{Agents:} RES (Research), GEN (Generation), PAT (Patching), CVA (Coverage Analysis), REF (Refinement). Iter = ReAct Iterations, Tools = Tool Calls. Multiple values (e.g., 5/10) indicate successive rounds. 
\end{flushleft}
\vspace{-3mm}
\label{tab:generation-cost}
\end{table}

%
Table~\ref{tab:generation-cost} details the computational costs and agent activity patterns across the five main targets. Harness generation costs range from \$1.34 (guava) to \$6.25 (commons-cli), with an average of \$3.20 per harness. Generation completes in an average of 599 seconds (~10 minutes), making the approach practical for integration into iterative fuzzing workflows.
Token consumption directly correlates with workflow complexity. The observed diversity in workflow iterations and agent loops indicates effective adaptation to target-specific characteristics, with more complex APIs requiring additional research and refinement cycles while simpler targets lead to quicker convergence. 
\par
Comparing usage patterns between the research agent and the generation agent reveals that the initial report contains most of the necessary information for synthesis requiring on average only $3.1$ iteration until the first harness is synthesized. Additionally, the patching agent invocation pattern reveals that, in most cases, the initial synthesis is already correct and only few targets require repair iterations with all evaluated harnesses compiling successfully after at most $6$ iterations.
The coverage analysis and refinement agents show high variance reflecting target-specific characteristics.
This adaptive behavior demonstrates that our agent-based termination successfully distinguishes between targets where refinement yields benefits and those where additional iteration would waste resources.




\section{Related Work}
\label{sec:related-work}

This section reviews existing approaches to automated harness generation, positioning our work within the broader research landscape through a comprehensive survey of classical and modern techniques.

\subsection{Classical Approaches to Harness Generation}
\label{subsec:classical-harness-generation}



Different traditions of program analysis have shaped how fuzzing harnesses are constructed. A first line of work can be described as \textbf{(1) usage-based generation}, where valid API calls are mined from existing consumer code or unit tests and then repurposed to build drivers. This strategy captures realistic interaction patterns between libraries and clients, as demonstrated by systems that slice client code into reusable API snippets~\cite{DBLP:conf/sigsoft/BabicBCIKKLSW19:FUDGE}, construct harness stubs from API dependence graphs extracted from consumer projects~\cite{DBLP:conf/uss/IspoglouAMP20:FuzzGen}, inject fuzzed inputs into test cases while preserving their order~\cite{DBLP:conf/sp/JeongJYMKJKSH23:UTopia}, or aggregate traces and snippets from package repositories~\cite{DBLP:journals/pacmse/WuNH25:WildSync, DBLP:conf/icse/ZhangZLWLZJ23:Daisy}.

In contrast, \textbf{(2) structure- or API-based generation} avoids reliance on external code and derives harnesses directly from type signatures, header files, or interface specifications. Approaches in this category focus on risky functions such as dereferences~\cite{DBLP:conf/icse:ZhangLMZ021:IntelliGen}, build dataflow graphs to capture API interactions~\cite{DBLP:conf/icse/GreenA22:GraphFuzz}, adapt this principle to specific domains such as OEM Android components and JNI bindings~\cite{DBLP:conf/ccs/ChenXLWC23:Hopper, DBLP:conf/uss/IspoglouAMP20:FuzzGen, DBLP:conf/issta/XiongDCQWSZ24:Atlas}, or introduce intermediate representations, API-flow clustering, and constraint learning to enable application to large-scale or closed-source libraries~\cite{DBLP:journals/pacmse/ToffaliniBTP25:LibErator, DBLP:conf/icse/ShermanN25:OGHarn, DBLP:conf/ndss/0007ZLSZLQ25:NEXZZER}.

Finally, there is \textbf{(3) evolutionary or feedback-driven generation}, where harnesses are refined iteratively based on runtime signals such as coverage or oracle checks. This approach uses automaton learning on API usage patterns~\cite{DBLP:conf/uss/ZhangLZZZZXLL0H23:Rubick}, ranks and selects candidate harnesses after short fuzzing trials~\cite{DBLP:journals/pacmse/ToffaliniBTP25:LibErator}, filters out misuses through relation learning~\cite{DBLP:conf/ndss/0007ZLSZLQ25:NEXZZER}, or validates candidates using compile-time and runtime oracles~\cite{DBLP:conf/icse/ShermanN25:OGHarn}. These systems demonstrate how feedback loops can drive continuous improvement, in contrast to static one-shot generation.

\subsection{LLM-based Approaches}
\label{subsec:llm-approaches}


With the availability of large language models, harness generation has also been explored from a learning perspective. One stream of work takes the form of \textbf{(1) feasibility studies}, which probe the capabilities and limitations of LLMs in fuzzing. These studies evaluate different prompting strategies across benchmarks~\cite{DBLP:conf/issta/ZhangZBLMXLSL24:HowEffectiveAreThey} and discuss fundamental obstacles such as semantic drift and the absence of reliable oracles~\cite{DBLP:conf/sigsoft/Jiang0MCZSWFWLZ24:WhenFuzzingMeetsLLMs}.

Beyond these exploratory efforts, a growing body of research investigates \textbf{(2) automatic harness synthesis with LLMs}. Systems in this category introduce coverage-guided prompt mutation for iterative refinement~\cite{DBLP:conf/ccs/LyuXCC24:PromptFuzz}, augment LLM reasoning with knowledge graphs of API relations~\cite{DBLP:conf/icse/XuMZZCHLW25:CKGFuzzer}, address binary-only interfaces with agent-based synthesis~\cite{DBLP:journals/corr/abs-2507-15058:LibLMFuzz}, target unsafe Rust APIs~\cite{DBLP:journals/corr/abs-2506-15648:deepSURF}, integrate CVE metadata to guide call-chain harnesses~\cite{DBLP:journals/corr/abs-2505-03425:HGFuzzer}, or employ constraint dependency graphs for Python libraries~\cite{DBLP:journals/cybersec/LiuLZZLL25:LLM4TDG}.

A particularly promising movement is \textbf{(3) hybrid and multi-agent architectures}, where LLMs are embedded into broader analytic ecosystems. Approaches in this category orchestrate several agents around a knowledge graph~\cite{DBLP:conf/icse/XuMZZCHLW25:CKGFuzzer}, combine LLM-based repair with solver-driven scheduling~\cite{DBLP:journals/corr/abs-2507-18289:Scheduzz}, or integrate LLM reasoning into static analysis pipelines built on CodeQL and Tree-Sitter~\cite{DBLP:journals/corr/abs-2505-03425:HGFuzzer}. These works illustrate the potential of combining generative capabilities with structured analysis and feedback mechanisms.

\subsection{Research Gaps}
\label{subsec:research-gaps}

Taken together, the literature reveals clear strengths and weaknesses across approaches. In the classical camp, \textbf{(1) usage-based methods} excel at extracting realistic API sequences but remain dependent on the availability of suitable consumer code~\cite{DBLP:conf/sigsoft/BabicBCIKKLSW19:FUDGE, DBLP:conf/uss/IspoglouAMP20:FuzzGen, DBLP:conf/sp/JeongJYMKJKSH23:UTopia}. Meanwhile, \textbf{(2) structure- or API-based methods} offer broader applicability but often lack iterative refinement, leaving adaptation to new environments largely manual~\cite{DBLP:conf/icse/GreenA22:GraphFuzz, DBLP:conf/ccs/ChenXLWC23:Hopper, DBLP:conf/ndss/0007ZLSZLQ25:NEXZZER}. Even \textbf{(3) feedback-driven systems} show promise yet frequently rely on domain-specific heuristics or narrow evaluation loops~\cite{DBLP:conf/uss/ZhangLZZZZXLL0H23:Rubick, DBLP:conf/icse/ShermanN25:OGHarn}.

LLM-based approaches display complementary gaps. While \textbf{(1) feasibility studies} clarify potential, they stop short of operational workflows~\cite{DBLP:conf/issta/ZhangZBLMXLSL24:HowEffectiveAreThey, DBLP:conf/sigsoft/Jiang0MCZSWFWLZ24:WhenFuzzingMeetsLLMs}. Systems for \textbf{(2) automatic synthesis} demonstrate strong results but tend to remain domain-bound~\cite{DBLP:conf/ccs/LyuXCC24:PromptFuzz, DBLP:journals/corr/abs-2507-15058:LibLMFuzz, DBLP:journals/corr/abs-2506-15648:deepSURF, DBLP:journals/cybersec/LiuLZZLL25:LLM4TDG}. And although \textbf{(3) hybrid and multi-agent solutions} represent an important step forward, very few manage to integrate generation, repair, fuzzing, and optimization into a fully automated feedback loop~\cite{DBLP:conf/icse/XuMZZCHLW25:CKGFuzzer, DBLP:journals/corr/abs-2507-18289:Scheduzz, DBLP:journals/corr/abs-2505-03425:HGFuzzer}.

Our work addresses this gap by presenting a multi-agent system tailored for Java libraries, where harnesses are generated, repaired, and improved iteratively under explicit coverage guidance. By providing an end-to-end pipeline—from Maven dependency resolution to fuzzing and coverage analysis—it unites structural analysis with adaptive learning and demonstrates significant improvements in achieved coverage compared to both manual and generic baselines.
\begin{itemize}
    \item We build on the strengths of structure-based generation to ensure broad applicability across Java libraries, while avoiding dependence on external code.
    \item We integrate LLM-based synthesis with static analysis and feedback-driven refinement, creating a closed-loop system that continuously improves harness quality.
    \item We demonstrate the effectiveness of our approach through extensive evaluation on real-world Java libraries, showing substantial coverage gains over existing methods.
\end{itemize}

\section{Conclusion and Outlook}

We presented a multi-agent architecture that integrates LLM reasoning with code analysis to fully automate fuzzing harness generation for Java libraries. By combining query-driven information retrieval through the Model Context Protocol with method-targeted coverage instrumentation and agent-guided termination, our approach achieves competitive coverage with manually-written OSS-Fuzz baselines while discovering real vulnerabilities in production libraries. With generation costs averaging \$3.20 and approximately 10 minutes per harness, the approach demonstrates practical feasibility for integration into continuous fuzzing workflows.

%\subsection{Threats to Validity}%
\label{subsec:threats}

\paragraph{Internal Validity}
LLM non-determinism poses challenges despite temperature=0 configuration, as Claude may produce varying outputs across runs. We mitigate this through convergence detection via code hashing and iteration limits, but different agent trajectories could yield different final harnesses with potentially different coverage profiles.

Callgraph depth configuration affects reachability analysis: we use depth-10 by default (depth-5 for large libraries like guava). Shallower depths may miss reachable methods, while deeper analysis increases agent context size and token costs. Our choice balances completeness against practical constraints.

Coverage sampling at 60-second intervals provides sufficient granularity for 12-hour campaigns, though higher-frequency sampling might reveal transient coverage dynamics. Branch coverage typically exhibits monotonic growth, limiting concerns about missed coverage spikes.

\paragraph{External Validity}
Our benchmark suite covers 7 methods across 5-6 libraries, representing diverse API patterns (parsers, serializers, utilities, generators). While these targets span common library categories, larger-scale evaluation across hundreds of targets would strengthen generalization claims about agent behavior and success rates.

The approach specifically targets JVM libraries with Maven dependency management and Javadoc documentation. Adaptation to other ecosystems (Python with pip, Rust with cargo, Go modules) requires porting static analysis tooling, documentation parsers, and build system integration. The core multi-agent architecture and MCP-based tool access should transfer, but ecosystem-specific implementation represents non-trivial engineering effort.

We exclusively use Claude 4.5 Sonnet as the underlying model. Different LLMs (GPT-4, Llama 3, Gemini) exhibit different code synthesis capabilities, reasoning patterns, and cost profiles. Future work should validate that our agent architecture and prompt design generalize across model families.

Computational constraints limited us to single 12-hour fuzzing runs per configuration. Multiple runs with different random seeds and statistical testing (e.g., Mann-Whitney U test) would provide stronger confidence in coverage comparisons and better characterize variance in fuzzing outcomes.

\paragraph{Construct Validity}
Branch coverage serves as a proxy for harness quality but doesn't guarantee bug-finding capability. A harness achieving 90\% coverage may still miss critical edge cases triggering vulnerabilities, while a 60\% coverage harness might trigger specific bugs through focused input patterns. We use coverage as a standardized metric enabling comparison across approaches, acknowledging its limitations.

Method-targeted coverage instrumentation improves alignment with target-specific fuzzing goals but may undercount relevant coverage in utility methods or deeply nested call chains invoked by the target. This trade-off prioritizes focused metrics over comprehensive library coverage.

Cost measurements include only LLM API charges (tokens × price per token) but exclude infrastructure costs: callgraph construction, documentation parsing, and build environment preparation. Total deployment cost in production would be higher, though these fixed costs amortize across multiple targets within the same library ecosystem.


%%
%% The next two lines define the bibliography style to be used, and
%% the bibliography file.
\bibliographystyle{ACM-Reference-Format}
\bibliography{main}


\end{document}
\endinput
%%
%% End of file `sample-sigconf.tex'.