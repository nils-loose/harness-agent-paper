\documentclass[sigconf,review,anonymous]{acmart}

\usepackage{listings}
\usepackage[inkscapeopt={-D}]{svg}
\usepackage{booktabs}
\usepackage{multirow}
\usepackage{pifont}
\usepackage{rotating}
\usepackage{listings}
\usepackage{xcolor}

\usepackage{placeins}
% YAML syntax highlighting - clean for academic papers
\lstdefinestyle{yaml}{
  basicstyle=\ttfamily\small,
  sensitive=false,
  comment=[l]{\#},
  commentstyle=\color{gray},
  showstringspaces=false,
  emphstyle=\bfseries,
  emph={get_method_code,description,method_signature,type,class_name,default},
  identifierstyle=,
  keywords={null, string},
}

% Java syntax highlighting
\definecolor{javared}{rgb}{0.6,0,0}          % for strings
\definecolor{javagreen}{rgb}{0.25,0.5,0.35}  % for comments
\definecolor{javapurple}{rgb}{0.5,0,0.35}    % for keywords
\definecolor{javadocblue}{rgb}{0.25,0.35,0.75} % for javadoc

\lstset{language=Java,
  basicstyle=\ttfamily\small,
  keywordstyle=\color{javapurple}\bfseries,
  stringstyle=\color{javared},
  commentstyle=\color{javagreen},
  morecomment=[s][\color{javadocblue}]{/**}{*/},
  showstringspaces=false,
  tabsize=2
}

% Define checkmark symbol
\newcommand{\cmark}{\ding{51}}

\setcopyright{acmlicensed}
\copyrightyear{2018}
\acmYear{2018}
\acmDOI{XXXXXXX.XXXXXXX}
\acmConference[SBFT '26]{SEARCH-BASED AND FUZZ TESTING}{April 12--18,
  2026}{Rio de Janeiro, Brazil}
\acmISBN{978-1-4503-XXXX-X/2026/04}
\acmSubmissionID{123}
% Define text styles for SVG rendering
\def\axtext{\footnotesize\sffamily}
\def\ticktext{\scriptsize\sffamily}


\begin{document}

\title{Coverage-Guided Multi-Agent Harness Generation for Java Library Fuzzing}

%%
%% The "author" command and its associated commands are used to define
%% the authors and their affiliations.
%% Of note is the shared affiliation of the first two authors, and the
%% "authornote" and "authornotemark" commands
%% used to denote shared contribution to the research.
\author{Ben Trovato}
\authornote{Both authors contributed equally to this research.}
\email{trovato@corporation.com}
\orcid{1234-5678-9012}
\author{G.K.M. Tobin}
\authornotemark[1]
\email{webmaster@marysville-ohio.com}
\affiliation{%
  \institution{Institute for Clarity in Documentation}
  \city{Dublin}
  \state{Ohio}
  \country{USA}
}

\author{Lars Th{\o}rv{\"a}ld}
\affiliation{%
  \institution{The Th{\o}rv{\"a}ld Group}
  \city{Hekla}
  \country{Iceland}}
\email{larst@affiliation.org}

\author{Valerie B\'eranger}
\affiliation{%
  \institution{Inria Paris-Rocquencourt}
  \city{Rocquencourt}
  \country{France}
}

\author{Aparna Patel}
\affiliation{%
 \institution{Rajiv Gandhi University}
 \city{Doimukh}
 \state{Arunachal Pradesh}
 \country{India}}

\author{Huifen Chan}
\affiliation{%
  \institution{Tsinghua University}
  \city{Haidian Qu}
  \state{Beijing Shi}
  \country{China}}

\author{Charles Palmer}
\affiliation{%
  \institution{Palmer Research Laboratories}
  \city{San Antonio}
  \state{Texas}
  \country{USA}}
\email{cpalmer@prl.com}

\author{John Smith}
\affiliation{%
  \institution{The Th{\o}rv{\"a}ld Group}
  \city{Hekla}
  \country{Iceland}}
\email{jsmith@affiliation.org}

\author{Julius P. Kumquat}
\affiliation{%
  \institution{The Kumquat Consortium}
  \city{New York}
  \country{USA}}
\email{jpkumquat@consortium.net}

\renewcommand{\shortauthors}{Trovato et al.}


\begin{abstract}
Coverage-guided fuzzing has proven effective for discovering vulnerabilities, but targeting library code requires specialized fuzz harnesses that translate fuzzer-generated inputs into valid API invocations. Manual harness creation is time-consuming and expertise-intensive, requiring deep understanding of API semantics, initialization sequences, and exception handling contracts.
We present a multi-agent architecture that automates fuzz harness generation for Java libraries through specialized LLM-powered agents. Five ReAct agents decompose the workflow into research, synthesis, compilation repair, coverage analysis, and refinement. Rather than preprocessing entire codebases, agents query documentation, source code, and callgraph information on demand through the Model Context Protocol, maintaining focused context while exploring complex dependencies.
To enable effective refinement, we introduce method-targeted coverage that tracks coverage only during target method execution to isolate target behavior, and agent-guided termination that examines uncovered source code to distinguish productive refinement opportunities from diminishing returns.
We evaluated our approach on seven target methods from six widely-deployed Java libraries totaling 115,000+ Maven dependents. Our generated harnesses achieve a median 26\% improvement over manually-written OSS-Fuzz baselines and outperform Jazzer AutoFuzz by 5\% in package-scope coverage. Generation costs average \$3.20 and 10 minutes per harness, making the approach practical for continuous fuzzing workflows. During 12-hour fuzzing campaigns, our harnesses discovered 3 unreported bugs in methods that are already integrated into OSS-Fuzz, demonstrating the effectiveness of the generated harnesses.
\end{abstract}
%%
%% The code below is generated by the tool at http://dl.acm.org/ccs.cfm.
%% Please copy and paste the code instead of the example below.
%%
\begin{CCSXML}
<ccs2012>
   <concept>
       <concept_id>10010147.10010257</concept_id>
       <concept_desc>Computing methodologies~Machine learning</concept_desc>
       <concept_significance>500</concept_significance>
       </concept>
   <concept>
       <concept_id>10011007</concept_id>
       <concept_desc>Software and its engineering</concept_desc>
       <concept_significance>500</concept_significance>
       </concept>
   <concept>
       <concept_id>10011007.10011006</concept_id>
       <concept_desc>Software and its engineering~Software notations and tools</concept_desc>
       <concept_significance>500</concept_significance>
       </concept>
   <concept>
       <concept_id>10002978.10003022</concept_id>
       <concept_desc>Security and privacy~Software and application security</concept_desc>
       <concept_significance>500</concept_significance>
       </concept>
 </ccs2012>
\end{CCSXML}

\ccsdesc[500]{Computing methodologies~Machine learning}
\ccsdesc[500]{Software and its engineering}
\ccsdesc[500]{Software and its engineering~Software notations and tools}
\ccsdesc[500]{Security and privacy~Software and application security}

%%
%% Keywords. The author(s) should pick words that accurately describe
%% the work being presented. Separate the keywords with commas.
\keywords{fuzzing, fuzz harness generation, large language models, multi-agent systems, automated software testing, program analysis, feedback-driven refinement, Java ecosystem, coverage-guided fuzzing, software security}


\maketitle


\section{Introduction}

Coverage-guided fuzzing has become a fundamental technique for discovering bugs and vulnerabilities in software systems. When applied to library code, its effectiveness depends on the availability of high-quality fuzz harnesses. A harness serves as an adapter between the fuzzer and the target library, transforming unstructured byte sequences into valid API invocations that exercise library logic. Manual harness creation remains a significant obstacle to widespread fuzzing adoption. Developers must understand API contracts, construct valid object states, synthesize realistic call sequences, and implement appropriate exception handling. This time-intensive process limits the number of library APIs that receive comprehensive fuzzing coverage. This challenge is particularly acute for Java libraries, which remain underrepresented in continuous fuzzing infrastructure and research despite the widespread deployment of Java applications in production systems. 

Existing automated harness generation approaches face complementary limitations. Usage-based methods mine API interaction patterns from consumer code~\cite{DBLP:conf/sigsoft/BabicBCIKKLSW19:FUDGE,DBLP:conf/sp/JeongJYMKJKSH23:UTopia} but require access to substantial client corpora that may be unavailable for specialized or newly released libraries. Structure-based approaches derive harnesses from type signatures and interface specifications~\cite{DBLP:conf/icse/GreenA22:GraphFuzz,DBLP:conf/icse/ShermanN25:OGHarn} but struggle with implicit preconditions and often rely on domain-specific heuristics that limit generalizability. Feedback-driven methods employ iterative refinement based on runtime signals~\cite{DBLP:conf/uss/ZhangLZZZZXLL0H23:Rubick} but typically apply fixed termination thresholds without semantic interpretation of coverage gaps. Recent LLM-based systems demonstrate progress through coverage-guided prompt evolution~\cite{DBLP:conf/ccs/LyuXCC24:PromptFuzz} and knowledge graph augmentation~\cite{DBLP:conf/icse/XuMZZCHLW25:CKGFuzzer}. However, these approaches lack mechanisms for iterative, query-driven exploration of codebases during generation. 

Large language models present new opportunities for automated harness generation by combining code synthesis capabilities with domain specific knowledge. However, directly applying LLMs to this task introduces distinct challenges. Preprocessing entire API surfaces exhausts available context windows, particularly for large library ecosystems with extensive dependency graphs. One-shot generation fails on complex libraries requiring multi-step initialization sequences or non-obvious preconditions. Coverage-based refinement risks semantic drift when agents lack mechanisms to interpret what coverage gaps represent and whether they indicate addressable deficiencies or fundamental limitations. These observations suggest that effective harness generation requires an approach that integrates LLM reasoning with targeted program analysis, retrieves information on demand rather than preprocessing entire codebases, and interprets coverage feedback semantically rather than applying fixed numerical thresholds.

We address these challenges through a multi-agent architecture that decomposes harness generation into specialized reasoning tasks. Five ReAct agents handle distinct phases of the workflow. The research agent explores API documentation and source code to understand target method semantics. The synthesis agent transforms this understanding into initial harness implementations. The compilation agent diagnoses and repairs build errors through iterative refinement. The coverage analysis agent interprets coverage gaps by examining uncovered source code to determine whether further refinement is worthwhile. The refinement agent modifies harnesses to address identified coverage deficiencies. Rather than preprocessing entire API surfaces into static knowledge graphs, agents query information on demand through the Model Context Protocol. This query-driven approach retrieves documentation for specific methods, source code for particular classes, and callgraph fragments rooted at selected invocations. The design maintains focused context while exploring large dependency graphs, enabling autonomous handling of complex build configurations and iterative compilation repair.

Two technical mechanisms enable effective coverage-guided refinement. First, we introduce method-targeted coverage instrumentation that activates JaCoCo tracking only during target method execution. Standard instrumentation measures all executed code, creating misaligned incentives where harnesses invoke unrelated utility methods to inflate coverage metrics. Our approach ensures that coverage measurements reflect target behavior rather than incidental framework initialization. Second, we implement agent-guided termination that interprets coverage gaps through source code analysis. The coverage analysis agent examines uncovered methods to distinguish addressable deficiencies such as missing input variants or unexplored API paths from fundamental limitations such as unreachable defensive code or external I/O dependencies. This interpretation enables the system to stop refinement when it yields diminishing returns while continuing when concrete improvement strategies exist.

We evaluate our approach on seven target methods from six widely deployed Java libraries spanning parsers, JSON processors, and core utilities. The selected libraries total over 115,000 Maven dependents and represent real-world fuzzing targets. All targets have existing manually written harnesses in OSS-Fuzz, providing strong baselines for comparison. Our generated harnesses achieve a median improvement of 26\% in method-targeted coverage over OSS-Fuzz baselines and outperform both OSS-Fuzz and Jazzer AutoFuzz by 6\% and 5\% respectively under full package-scope coverage. Generation costs average \$3.20 and approximately 10 minutes per harness, demonstrating practical feasibility for integration into continuous fuzzing workflows. During 12-hour fuzzing campaigns, our harnesses discovered 3 previously unreported bugs in commons-cli and jsoup. These discoveries occurred in methods already covered by existing OSS-Fuzz harnesses, demonstrating that automated generation achieves sufficient quality for real vulnerability discovery. In one notable case involving ANTLR4, our generated harness achieved measurable coverage on a target method where the existing OSS-Fuzz harness registered 0\% coverage throughout the entire campaign due to inability to satisfy method preconditions.


In summary, this work makes the following contributions.
\begin{itemize}
    \item A multi-agent architecture that integrates LLM reasoning with program analysis to automate harness generation for Java library APIs without requiring consumer code corpora or manual intervention.

    \item A query-driven tool interface using the Model Context Protocol that retrieves precisely scoped program information on demand, preventing context saturation while enabling exploration of large codebases.

    \item Method-targeted coverage instrumentation with agent-guided termination that interprets coverage gaps through source code analysis rather than applying fixed numerical thresholds.

    \item Empirical validation on widely deployed libraries demonstrating competitive coverage with manually written baselines, practical generation costs, and discovery of three previously unreported bugs in production code.
\end{itemize}

\begin{figure*}[t]
    \centering
    \includesvg[width=.9\linewidth]{data/overview.svg}
    \caption{Schematic overview of the harness generation workflow. Agents are ReAct agents with specialized tool access.}
    \label{fig:overview}
\end{figure*}

\section{Preliminaries}

\paragraph{Harness Design for Coverage-Guided Fuzzing}
Coverage-guided fuzzers generate test inputs by mutating input sequences based on code coverage feedback. When fuzzing library APIs, a \emph{fuzz harness} serves as the entry point that translates fuzzer-generated bytes into valid API invocations. An effective harness must parse input bytes into appropriate data types, construct required object state to satisfy API preconditions, invoke target methods with derived arguments, and handle exceptions appropriately to distinguish expected error conditions from genuine bugs. The harness design directly impacts fuzzing effectiveness. A harness that immediately throws exceptions on most inputs wastes fuzzing cycles without exploring library logic. Conversely, a harness that exercises diverse API paths and satisfies complex preconditions enables the fuzzer to reach deeper code and discover latent bugs.

\paragraph{Tool-Augmented Reasoning}
Tool-augmented reasoning equips LLMs with external capabilities they can invoke during generation~\cite{DBLP:conf/nips/SchickDDRLHZCS23:Toolformer,DBLP:conf/iclr/QinLYZYLLCTQZHT24:ToolLLM}. The ReAct (Reasoning and Acting) paradigm~\cite{DBLP:conf/iclr/YaoZYDSN023:ReAct} formalizes tool use as an interleaved process where the model alternates between reasoning steps that generate natural language explanations of its strategy and action steps that invoke tools and observe outputs. For code generation, tools typically include documentation search, source code retrieval, compilation, and test execution. Rather than providing all information upfront in a single prompt, ReAct agents query information on demand as their reasoning progresses.

\paragraph{Multi-Agent LLM Architectures}
Recent work has demonstrated that LLMs can be organized into multi-agent systems where distinct instances specialize in complementary subtasks~\cite{DBLP:journals/corr/abs-2312-13010:AgentCoder}. Rather than relying on a single monolithic prompt, multi-agent architectures decompose complex objectives into stages handled by specialized agents that communicate through structured message passing or shared state. For code generation tasks, this decomposition enables separation of concerns where one agent explores API documentation, another synthesizes implementations, and a third diagnoses compilation errors. This division of labor has been shown to improve both generation quality and success rates on challenging benchmarks~\cite{DBLP:conf/iclr/0036YZXLL0DMYZ024:AgentBench}.

\paragraph{Model Context Protocol}
The Model Context Protocol~\cite{mcp} standardizes how LLM agents access external resources through structured tool interfaces. MCP defines a client-server architecture where agents act as clients that invoke tools exposed by servers managing data sources. Each tool accepts structured parameters and returns formatted responses optimized for LLM consumption. For code analysis tasks, an MCP server might expose tools such as retrieving method source code or searching API documentation. MCP enables query-driven information retrieval where agents request precisely scoped information as needed. 

\section{Automated Harness Generation}%
\label{sec:approach}
We present an agent-based approach to automated fuzzing harness generation that addresses several fundamental challenges of generating effective harnesses for complex library APIs. Our approach combines LLM-powered agents with static analysis and dynamic coverage feedback to iteratively construct and refine harnesses that achieve deep code coverage.
Figure~\ref{fig:overview} illustrates our workflow as a sequence of transformations that progressively refine a fuzzing harness. After initializing the environment by downloading library artifacts and preparing analysis infrastructure, the workflow proceeds through three phases: target research to understand API semantics, harness construction through code generation and compilation, and iterative coverage-guided refinement. Specialized ReAct agents~\cite{DBLP:conf/iclr/YaoZYDSN023:ReAct} orchestrate each phase, querying documentation and source code on demand to maintain focused context while exploring large dependency graphs.
\subsection{Static Analysis and Instrumentation}%
\label{subsec:static-analysis}
\begin{lstlisting}[language=Java,caption={Conceptual decompiled harness showing selective coverage instrumentation through runtime control of JaCoCo's recording state. The coverage handling calls are added using offline instrumentation after compilation.},label={lst:coverage-tracking},float=htbp,basicstyle=\small\ttfamily,numbers=left,numberstyle=\tiny\color{gray},frame=single]
public static void fuzzerTestOneInput(
      FuzzedDataProvider data) {
   RT.getAgent().setRecording(false);

   // Parameter/ Instance preparation
   Options o = prepareOptions(data);

   try{
      RT.getAgent().setRecording(true);
      Parser.parse(o);
      RT.getAgent().setRecording(false);
   } catch (IllegalArgumentException var15) {
        // Expected exception 
      }
}
\end{lstlisting}
Our approach requires three preprocessing artifacts that enable efficient agent exploration and accurate coverage measurement. First, we extract API documentation from Javadoc HTML archives distributed with Maven artifacts, parsing method signatures and parameter descriptions using Beautiful Soup~\cite{beautifulsoup} to provide agents with concise API contracts indexed for query-based retrieval. Second, we compute a static callgraph rooted at the target method using SootUp's Class Hierarchy Analysis~\cite{DBLP:conf/tacas/KarakayaSKBSLH24:SootUp}, traversing method invocations to depth~10 (depth~5 for large libraries). Each node records the method signature, enclosing class, and distance from the target, serving to scope coverage analysis to reachable methods and provide agents with structural context about call dependencies. Third, we implement method-targeted coverage instrumentation. Standard coverage instrumentation measures all executed code, creating a misaligned incentive for agents to invoke unrelated utility methods. We address this by extending JaCoCo~\cite{jacoco} with runtime toggling of coverage tracking, accessible through a runtime API. Using ASM~\cite{asm} for offline bytecode instrumentation, we wrap target method invocations to enable coverage recording only during target execution (Listing~\ref{lst:coverage-tracking}), ensuring metrics reflect the target's behavior rather than incidental framework initialization.
%
%
\subsection{Tool-Augmented Exploration}%
\label{subsec:tool-augmented-exploration}
\begin{figure}[t]
    \centering
    \includesvg[width=.9\linewidth]{data/mcp-design.svg}
    \caption{Overview of exposed tools through the model context protocol (MCP).}
    \label{fig:mcp-design}
\end{figure}
%
We expose preprocessing artifacts to agents through a query-based interface using the Model Context Protocol (MCP)~\cite{mcp}, enabling on-demand retrieval of tailored information as reasoning progresses.
%
%\paragraph{Tool Interface Design}
We provide three tool categories: documentation queries, source code retrieval, and callgraph queries. Figure~\ref{fig:mcp-design} illustrates the MCP initialization process. Each tool accepts structured parameters (e.g., class name, method signature) and returns responses optimized for LLM consumption: concise method signatures, minimal code snippets, and depth-limited callgraph fragments.
\begin{table}[t]
\caption{Tool availability across DRAG workflow agents. MCP tools are provided by three Model Context Protocol servers that enable LLM agents to query code intelligence APIs.}
\centering
\setlength{\tabcolsep}{3pt}
\renewcommand{\arraystretch}{1.1}
\small
\begin{tabular}{c l | c c c c c}
\toprule
& \multirow{2}{*}{\textbf{Tool}} & \multicolumn{5}{c}{\textbf{ReAct Agents}} \\
\cmidrule(lr){3-7}
& & \textbf{RSH} & \textbf{GEN} & \textbf{PAT} & \textbf{CVA} & \textbf{REF} \\
\midrule
\multirow{6}{*}{\rotatebox{90}{\textit{Docs}}}
& \texttt{method\_doc} & \cmark & \cmark & \cmark & \cmark & \cmark \\
& \texttt{class\_doc} & \cmark & \cmark & \cmark & \cmark & \cmark \\
& \texttt{package\_doc} & \cmark & \cmark & \cmark & \cmark & \cmark \\
& \texttt{list\_packages} & \cmark & \cmark & \cmark & \cmark & \cmark \\
& \texttt{list\_classes} & \cmark & \cmark & \cmark & \cmark & \cmark \\
& \texttt{list\_methods} & \cmark & \cmark & \cmark & \cmark & \cmark \\
\midrule
\multirow{5}{*}{\rotatebox{90}{\textit{Code}}}
& \texttt{get\_method\_code} & \cmark & \cmark & \cmark & \cmark & \cmark \\
& \texttt{get\_class\_code} & \cmark & \cmark & \cmark & \cmark & \cmark \\
& \texttt{find\_definition} & \cmark & \cmark & \cmark & \cmark & \cmark \\
& \texttt{find\_refs} & \cmark & \cmark & \cmark & \cmark & \cmark \\
& \texttt{grep} & \cmark & \cmark & \cmark & \cmark & \cmark \\
& \texttt{find\_symbol} & \cmark & \cmark & \cmark & \cmark & \cmark \\
\midrule
\multirow{2}{*}{\rotatebox{90}{\textit{CG}}}
& \texttt{reach\_methods} & -- & -- & -- & (\cmark) & -- \\
& \texttt{path\_to\_method} & -- & -- & -- & \cmark & -- \\
\midrule
\multirow{2}{*}{\rotatebox{90}{\textit{Exec}}}
& \texttt{compiler} & -- & -- & (\cmark) & -- & -- \\
& \texttt{jazzer} & -- & -- & -- & (\cmark) & -- \\
\bottomrule
\end{tabular}
\smallskip
\begin{flushleft}
\footnotesize
\textbf{Agents:} RSH (Research), GEN (Generation), PAT (Patching), CVA (Coverage Analysis), REF (Refinement).
\cmark~= Available, (\cmark)~= Static invocation (not queryable), --~= No access.
\textbf{Categories:} Docs (Javadoc API documentation); Code (source code indexing); Graph (call graph analysis); Exec (compiler and fuzzer).
\end{flushleft}
\vspace{-3mm}
\label{tab:tool-availability}
\end{table}
%\paragraph{Role-Based Tool Access}
To prevent exploration drift, we restrict tool access based on agent role. Table~\ref{tab:tool-availability} summarizes tool availability across the five ReAct agents. All agents access documentation and source code tools for foundational API understanding, while callgraph tools are restricted to the coverage analysis agent for interpreting coverage gaps. This role-based access control prevents agents from pursuing information irrelevant to their current task.

% Commented out to save space - MCP tool schema details not essential
% \begin{figure}[t]
\begin{lstlisting}[style=yaml,caption={MCP tool schema for the \texttt{get\_method\_code} tool from CodeContextMCP server. Tools are exposed to ReAct agents based on tag-based access control.},label={fig:mcp-tool-schema},numbers=left,numberstyle=\tiny\color{gray},frame=single]
get_method_code:
  description:
    Get source code for a specific method
  method_signature:
    type: string
    description:
      Source code method signature (e.g.,
      'String someMethod(String, int)' or
      'someMethod(String, int)'). If possible 
      provide full signature including parameters 
      for precise matching, however to get all 
      overloads 'someMethod' also works.
  class_name:
    type: string
    description:
      (Fully qualified) class name (e.g.,
      'com.example.Util'). If possible, use the
      fully qualified name. Required for accurate
      matching!
    default: null
\end{lstlisting}
\end{figure}
%
\subsection{Target Research}%
\label{subsec:target-research}

Following environment initialization (Maven download, documentation extraction, callgraph construction), the research agent transforms the target method signature into contextual knowledge about API semantics. The agent is initialized with the target method's signature, documentation, and source code, then iteratively queries additional documentation and source code through MCP tools. Figure~\ref{fig:research-sequence} illustrates this query-driven exploration pattern. Rather than exhaustively extracting all available information, the agent follows its reasoning to identify relevant patterns: required initialization sequences, factory method usage, and implicit preconditions. The agent produces a natural language research report structured with predefined markdown sections that organize findings without constraining content to rigid schemas, accommodating diverse API designs.
\begin{figure}[t]
\centering
\includesvg[width=.9\linewidth]{data/tool-usage-sequence.svg}
\caption{Sequence diagram showing initial tool interactions during the research phase for \texttt{Jsoup.parse(String)}. The researcher agent iteratively queries Javadoc MCP (for API documentation) and CodeContext MCP (for source code) to understand the target method's invocation requirements. Solid arrows represent tool calls; dashed arrows represent responses.}
\label{fig:research-sequence}
\end{figure}
\subsection{Harness Generation}%
\label{subsec:harness-generation}

The research report is transformed into compilable code through two sequential steps: generation and compilation. The generation agent synthesizes initial harness code that instantiates the target method with fuzzer-generated inputs. The agent has access to the Jazzer API documentation and queries additional source code to resolve ambiguities in constructor signatures or factory method usage. A critical aspect of harness synthesis is exception handling: the agent must determine which exceptions represent expected API behavior (e.g., IllegalArgumentException for invalid inputs) that should be caught to continue fuzzing, versus unexpected exceptions that indicate bugs and must propagate to Jazzer's crash detection. The agent analyzes API documentation and method signatures to infer expected exception contracts, synthesizing appropriate try-catch blocks that preserve bug-finding capability. The agent outputs harness source code and a list of Maven dependencies. Separating research from generation prevents context saturation: research explores broadly without committing to code structure, while generation focuses narrowly on producing syntactically valid harness code.

If the compile step fails, a compilation agent iteratively resolves build errors by analyzing compiler diagnostics, querying source code and documentation to understand the root cause, and producing corrected code until compilation succeeds or an iteration limit is reached. Common error patterns include missing imports, incorrect method signatures, and improper exception handling. Independent compilation repair allows targeted iteration budgets distinct from initial synthesis, reflecting the need for precise syntactic correctness in the final artifact.
\subsection{Coverage-Guided Refinement}%
\label{subsec:coverage-guided-refinement}

Once compilation succeeds, the compiled harness is instrumented and executed under fuzzing to collect initial coverage data. An iterative refinement loop then uses this coverage feedback to improve harness effectiveness through two collaborative agents: a coverage analysis agent that interprets coverage gaps and decides whether refinement is worthwhile, and a refinement agent that modifies the harness.

\paragraph{Coverage Analysis and Termination}
To seed the coverage analysis, we merge method-level coverage data with the static callgraph to produce an annotated view showing coverage status for each reachable method, grouped by call depth from the target. The coverage analysis agent explores uncovered or partially covered methods by querying their source code and documentation to determine whether gaps reflect addressable harness deficiencies (missing input diversity, unexplored API paths) or fundamental limitations (unreachable defensive code, external I/O dependencies). The agent then makes a termination decision: stop if further refinement yields diminishing returns, or continue with a strategy targeting specific uncovered methods.

\paragraph{Harness Refinement}
If refinement continues, the refinement agent receives the current harness code, the coverage analysis strategy (priority methods and improvement rationale), and annotated coverage data. The agent modifies the harness to exercise uncovered code paths through strategies such as diversifying input generation, invoking alternative API paths, or triggering exception handlers through edge-case inputs. The refined harness re-enters compilation and fuzzing, creating a feedback loop that continues until the coverage agent determines refinement yields diminishing returns or an iteration limit is reached. Convergence detection through code hashing prevents oscillation between semantically equivalent harness variants.

\section{Evaluation}%
\label{sec:evaluation}

We evaluate our harness generation approach on widely used Maven libraries, examining achieved coverage, computational costs, and agent behavior patterns. Our evaluation demonstrates that the approach produces competitive harnesses to existing baselines and techniques while maintaining practical generation costs. Additionally, during the 12-hour fuzzing campaigns, our harnesses uncovered multiple previously unknown bugs in mature libraries, validating their effectiveness in real-world testing scenarios.

\subsection{Experimental Setup}%
\label{subsec:exp-setup}
\begin{table*}[t]
\caption{Benchmark library characteristics. All libraries are widely-deployed in the Maven ecosystem and represent real-world fuzzing targets.}
\centering
\setlength{\tabcolsep}{3pt}
\renewcommand{\arraystretch}{1.1}
\small
\begin{tabular}{l r r l l l r}
\toprule
\textbf{Library} & \textbf{Version} & \textbf{Dependents} & \textbf{Class Name} & \textbf{Target Method} & \textbf{Category} & \textbf{Rank} \\
\midrule
commons-cli      & 1.10.0     & 5K+  & DefaultParser        & parse(Options, String[])              & CLI Parser       & \#1 \\
gson             & 2.13.1     & 27K+ & JsonParser           & parseString(String)                   & JSON Library     & \#2 \\
guava            & 33.4.8-jre & 42K+ & HostAndPort          & fromString(String)                    & Core Utilities   & \#1 \\
jackson-databind & 2.20.0     & 36K+ & ObjectMapper         & readTree(String)                      & JSON Library     & \#1 \\
jsoup            & 1.21.1     & 4K+  & Jsoup                & parse(String)                         & HTML Parser      & \#1 \\
\midrule
\multirow{2}{*}{antlr4} & \multirow{2}{*}{4.13.2} & \multirow{2}{*}{1K+} & Grammar & Grammar(String) & \multirow{2}{*}{Parser Generator} & \multirow{2}{*}{\#1} \\[-.9pt]
                &            &      & Grammar              & createParserInterpreter(TokenStream)  &                  &      \\
\bottomrule
\end{tabular}
\smallskip
\begin{flushleft}
\footnotesize
Dependents = Maven artifacts declaring this library as a dependency. Rank = Category ranking on Maven Central.
\end{flushleft}
\vspace{-3mm}
\label{tab:benchmarks}
\end{table*}

We evaluate on seven target methods from six widely-deployed Java libraries (Table~\ref{tab:benchmarks}). The selected targets span parsers (commons-cli, jsoup, antlr4), JSON libraries (gson, jackson-databind), and core utilities (guava).
%
%\paragraph{Baselines}
We compare our generated harnesses against two baselines. OSS-Fuzz~\cite{ossfuzz2017} is Google's continuous fuzzing service for open source software, launched in 2016 to provide large-scale fuzzing infrastructure. All selected target methods have existing harnesses in OSS-Fuzz. These harnesses serve as our primary baseline.
%
We additionally compare against Jazzer AutoFuzz~\cite{jazzer}, an automated harness generation mode built into the Jazzer coverage-guided fuzzer for the JVM. AutoFuzz leverages Java reflection to automatically generate harnesses by discovering accessible constructors and methods, recursively building required objects through structure-aware type instantiation. Unlike our approach, AutoFuzz operates without program analysis or coverage feedback, relying solely on runtime reflection to explore the API surface.
%
For LLM-based comparison, we attempted to use OSS-Fuzz-Gen~\cite{oss-fuzz-gen}, but encountered implementation issues, primarily in model output parsing, that prevented successful harness generation for our Java targets.
\par
We implement our approach using LangGraph~\cite{langgraph} for workflow orchestration and Claude 4.5 Sonnet (2025-09-29) as the underlying model. Harnesses are compiled using Gradle and executed using Jazzer with instrumented coverage collection. Our implementation and generated harnesses are available on GitHub.\footnote{Repository anonymized for submission.}
\par
To measure the effectiveness of fuzzing the targeted method, we measure coverage under two configurations: (1)~method-targeted coverage activates only during target method execution (Section~\ref{subsec:static-analysis}), focusing metrics on target behavior; (2)~full target-scope coverage uses standard JaCoCo instrumentation across the entire library for fair baseline comparison.
All campaigns are run for 12 hours per target with a single fuzzing thread and an empty seed corpus. 
\subsection{Coverage Effectiveness}%
\label{subsec:coverage-effectiveness}

\begin{figure*}[t]
    \centering
    % Row 1: Method-Targeted Coverage (Ours only)
    \textbf{Method-Targeted Coverage}\\[4pt]
    \begin{tabular}{ccccc}
        \includesvg[width=0.18\textwidth]{data/coverage/commons_coverage_comparison.svg} &
        \includesvg[width=0.18\textwidth]{data/coverage/gson_coverage_comparison.svg} &
        \includesvg[width=0.18\textwidth]{data/coverage/guava_coverage_comparison.svg} &
        \includesvg[width=0.18\textwidth]{data/coverage/jackson_coverage_comparison.svg} &
        \includesvg[width=0.18\textwidth]{data/coverage/jsoup_coverage_comparison.svg}
    \end{tabular}

    \vspace{2pt}

    % Row 2: Full Target-Scope Coverage (Ours vs AutoFuzz)
    \textbf{Full Target-Scope Coverage}\\[4pt]
    \begin{tabular}{ccccc}
        \includesvg[width=0.18\textwidth]{data/coverage/package_commons-cli_coverage.svg} &
        \includesvg[width=0.18\textwidth]{data/coverage/package_gson_coverage.svg} &
        \includesvg[width=0.18\textwidth]{data/coverage/package_guava_coverage.svg} &
        \includesvg[width=0.18\textwidth]{data/coverage/package_jackson-databind_coverage.svg} &
        \includesvg[width=0.18\textwidth]{data/coverage/package_jsoup_coverage.svg} \\
        commons-cli & gson & guava & jackson-databind & jsoup
    \end{tabular}

    \caption{Coverage comparison across five Java libraries over 12-hour fuzzing campaigns. \textbf{Top row:} Method-targeted coverage for our generated harnesses, focusing exclusively on target method execution. \textbf{Bottom row:} Full target-scope coverage enabling fair comparison with AutoFuzz baseline. Each plot shows branch coverage percentage over time.}
    \label{fig:coverage-comparison}
\end{figure*}

Figure~\ref{fig:coverage-comparison} shows line coverage over time across five Java libraries. The top row compares method-targeted coverage for our generated harnesses against the OSS-Fuzz baseline, evaluating focused execution of target method logic. The bottom row shows full package-scope coverage allowing comparison against AutoFuzz.
Under method-targeted coverage the generated harnesses have a median improvement of $26$\% over the OSS-Fuzz harnesses, demonstrating the effectiveness of system in generating harnesses for specific methods. The temporal dynamics reveal that the primary difference is observable early into the fuzzing campaign, suggesting that the harness provides better structural input diversity.
Comparing under full target-scope coverage, our generated harnesses outperform the AutoFuzz and OSS-Fuzz baseline by a median of $5$\% and $6$\% respectively. The only target the generated harness does not outperform the baselines is jackson-databind, where for the OSS-Fuzz harness the harness contains additional fuzzing logic after the execution of the target method increasing the overall coverage. 
%
%\paragraph{Coverage Trajectory Analysis}
%Beyond final coverage values, the temporal dynamics reveal important differences in how harnesses explore target code. Our harnesses typically reach 80\% of their final coverage within the first hour, indicating effective initial synthesis that exercises core API paths immediately. In contrast, AutoFuzz exhibits slower convergence, often requiring 2-3 hours to plateau. This suggests that reflection-based type instantiation discovers shallow API interactions quickly but struggles to synthesize complex call sequences required for deeper coverage.
%
%Notably, gson and jackson-databind show continued coverage growth throughout the 12-hour campaign for our harnesses, while AutoFuzz plateaus within 3-4 hours. This sustained exploration indicates that our harnesses exercise diverse input patterns that trigger different code paths, whereas AutoFuzz's deterministic type coercion limits input space diversity.
%
\par
\begin{figure}[t]
    \centering
    \begin{tabular}{cc}
        \includesvg[width=0.45\columnwidth]{data/coverage/antlr4-grammar_coverage_comparison.svg} &
        \includesvg[width=0.45\columnwidth]{data/coverage/antlr4-parser_coverage_comparison.svg} \\
        (a) Grammar(String) & (b) createParserInterpreter
    \end{tabular}
    \caption{Method-targeted coverage for ANTLR4's two target methods, comparing OSS-Fuzz harness variants against our generated harnesses.}
    \label{fig:antlr4-coverage}
\end{figure}

To demonstrate the benefits of targeted harness generation, we evaluate ANTLR4, which has two target methods in our benchmark. The existing OSS-Fuzz harness~\footnote{https://github.com/google/oss-fuzz/blob/master/projects/antlr4-java/GrammarFuzzer.java} exercises both methods sequentially: it creates a Grammar object from fuzzed input, then invokes createParserInterpreter on the resulting grammar. This sequential dependency means the parser interpreter is only reached when grammar creation succeeds without throwing exceptions.
We compare our automatically generated harnesses (each targeting one method individually) against the OSS-Fuzz harness measured with method-targeted coverage scoped to each target method. For the Grammar constructor (Figure~\ref{fig:antlr4-coverage}(a)), we evaluate both the unmodified OSS-Fuzz harness and a manually edited variant that only creates the grammar without invoking the parser interpreter. For createParserInterpreter (Figure~\ref{fig:antlr4-coverage}(b)), we evaluate only the unmodified OSS-Fuzz harness.
Our generated harnesses outperform the OSS-Fuzz baseline in both scenarios. Most notably, the OSS-Fuzz harness achieves 0\% coverage for createParserInterpreter throughout the campaign, indicating that grammar creation consistently throws exceptions before reaching the parser interpreter call. In contrast, our generated harness successfully exercises this method by synthesizing inputs that satisfy the grammar constructor's preconditions. Note that both ANTLR4 campaigns encountered a Jazzer timeout at 30 minutes that terminates execution; we were unable to configure Jazzer to continue, though the coverage trends before termination clearly demonstrate the performance difference.
%
\subsection{Bug Discovery}%
\label{subsec:bug-discovery}
Beyond coverage metrics, we examine whether our generated harnesses discover actual bugs during fuzzing campaigns. Jazzer reports crashes through its exception handling infrastructure, distinguishing between genuine crashes (uncaught exceptions indicating bugs) and caught exceptions that represent normal control flow. This presents are core challenge for the harness generation, as the generated harnesses must avoid over-catching exceptions that would mask real bugs while also prevent spurious crashes from expected error handling.
Across the 12-hour fuzzing campaigns, the generated harnesses triggered a total of 14 crashes in two libraries. After manual investigation we determine, that all reported crashes represent genuine bugs in the target library. The harnesses correctly identified uncaught exceptions rather than reporting false positives from expected exception handling. Manual triage revealed 3 unique bugs:

\begin{itemize}
\item \textbf{commons-cli}: Two distinct null pointer exceptions in option parsing logic, triggered by edge-case combinations of option configurations and malformed arguments. The crashes manifest in both long-option and short-option code paths, with 12 total crash artifacts reducing to 2 unique root causes.
\item \textbf{jsoup}: One index-out-of-bounds exception in HTML tree building logic, triggered by complex malformed HTML input (~1KB). This crash represents a potential denial-of-service vector, as attackers could craft HTML to crash the parser. Two crash artifacts correspond to the same underlying bug.
\end{itemize}

We are currently coordinating responsible disclosure with library maintainers. The commons-cli bugs represent robustness issues, while the jsoup bug has higher severity due to its potential impact on server-side HTML processing.
These results demonstrate that our automatically-generated harnesses achieve sufficient input diversity and API coverage to discover real bugs in mature, widely-deployed libraries that already have existing harnesses in the OSS-Fuzz ecosystem. The fact that Jazzer reported zero false positives highlights the effectiveness of the harness generation in balancing exception handling to expose genuine bugs without over-catching.
%
\subsection{Generation Costs and Agent Behavior}%
\label{subsec:generation-costs}
%
\begin{table}[t]
\caption{Harness generation cost and agent activity across target libraries. Agent rows show iterations and tool calls, with multiple values (e.g., 5/10) indicating successive refinement rounds.}
\centering
\setlength{\tabcolsep}{3pt}
\renewcommand{\arraystretch}{1.1}
\small
\begin{tabular}{l l | r r r r r | r}
\toprule
& \textbf{Metric} & \textbf{commons-cli} & \textbf{gson} & \textbf{guava} & \textbf{jackson} & \textbf{jsoup} & \textbf{Avg} \\
\midrule
& Tokens & 1896K & 557K & 395K & 1285K & 1039K & 982K \\
& Cost & \$6.25 & \$1.81 & \$1.34 & \$4.13 & \$3.32 & \$3.20 \\
& Time & 1200s & 337s & 417s & 684s & 551s & 599s \\
\midrule
\multirow{2}{*}{\rotatebox{90}{\textbf{RES}}} & Iter & 17 & 17 & 16 & 17 & 19 & 17.7 \\
& Tools & 44 & 27 & 23 & 24 & 28 & 30.7 \\
\midrule
\multirow{2}{*}{\rotatebox{90}{\textbf{GEN}}} & Iter & 3 & 2 & 3 & 3 & 4 & 3.1 \\
& Tools & 5 & 2 & 2 & 3 & 4 & 3.4 \\
\midrule
\multirow{2}{*}{\rotatebox{90}{\textbf{PAT}}} & Iter & 5/0/0/0 & 0 & 0/0 & 6/0 & 0/0 & 0.9 \\
& Tools & 11/0/0/0 & 0 & 0/0 & 6/0 & 0/0 & 1.2 \\
\midrule
\multirow{2}{*}{\rotatebox{90}{\textbf{CVA}}} & Iter & 16/14/13/14 & 17 & 4/4 & 18/14 & 14/18 & 12.5 \\
& Tools & 28/22/16/28 & 31 & 4/5 & 21/16 & 24/21 & 18.3 \\
\midrule
\multirow{2}{*}{\rotatebox{90}{\textbf{REF}}} & Iter & 10/7/15 & -- & 5 & 8 & 10 & 8.9 \\
& Tools & 19/15/23 & -- & 5 & 13 & 22 & 16.1 \\
\bottomrule
\end{tabular}
\smallskip
\begin{flushleft}
\footnotesize
\textbf{Agents:} RES (Research), GEN (Generation), PAT (Patching), CVA (Coverage Analysis), REF (Refinement). \\
Iter = Iterations, Tools = Tool Calls. Multiple values (e.g., 5/10) indicate successive refinement rounds. -- = no activity.
\end{flushleft}
\vspace{-3mm}
\label{tab:generation-cost}
\end{table}

%
Table~\ref{tab:generation-cost} details the computational costs and agent activity patterns across the five main targets. Harness generation costs range from \$1.34 (guava) to \$6.25 (commons-cli), with an average of \$3.20 per harness. Generation completes in an average of 599 seconds (~10 minutes), making the approach practical for integration into iterative fuzzing workflows.
Token consumption directly correlates with workflow complexity. The observed diversity in workflow iterations and agent loops indicates effective adaptation to target-specific characteristics, with more complex APIs requiring additional research and refinement cycles while simpler targets lead to quicker convergence. 
\par
Comparing usage patterns between the research agent and the generation agent reveals that the initial report contains most of the necessary information for synthesis requiring on average only $3.1$ iteration until the first harness is synthesized. Additionally, the patching agent invocation pattern reveals that, in most cases, the initial synthesis is already correct and only few targets require repair iterations with all evaluated harnesses compiling successfully after at most $6$ iterations.
The coverage analysis and refinement agents show high variance reflecting target-specific characteristics.
This adaptive behavior demonstrates that our agent-based termination successfully distinguishes between targets where refinement yields benefits and those where additional iteration would waste resources.




\section{Related Work}
\label{sec:related-work}

\paragraph{Classical Harness Generation}
Different traditions of program analysis have shaped how fuzzing harnesses are constructed. \textit{Usage-based generation} mines valid API calls from existing consumer code or unit tests, as demonstrated by systems that slice client code into reusable API snippets~\cite{DBLP:conf/sigsoft/BabicBCIKKLSW19:FUDGE}, construct harness stubs from API dependence graphs~\cite{DBLP:conf/uss/IspoglouAMP20:FuzzGen}, or inject fuzzed inputs into test cases~\cite{DBLP:conf/sp/JeongJYMKJKSH23:UTopia}. While this captures realistic interaction patterns, it remains dependent on the availability of suitable consumer code. In contrast, \textit{structure-based generation} derives harnesses directly from type signatures and interface specifications, building dataflow graphs to capture API interactions~\cite{DBLP:conf/icse/GreenA22:GraphFuzz} or introducing intermediate representations for large-scale libraries~\cite{DBLP:journals/pacmse/ToffaliniBTP25:LibErator, DBLP:conf/icse/ShermanN25:OGHarn}. These approaches offer broader applicability but often lack iterative refinement, leaving adaptation to new targets largely manual. \textit{Feedback-driven generation} refines harnesses iteratively using runtime signals, employing automaton learning on API usage patterns~\cite{DBLP:conf/uss/ZhangLZZZZXLL0H23:Rubick} or validating candidates using compile-time and runtime oracles~\cite{DBLP:conf/icse/ShermanN25:OGHarn}. While promising, such systems frequently rely on domain-specific heuristics or fixed coverage thresholds for termination decisions.

\paragraph{LLM-based Harness Synthesis}
% TODO: Add OSS-Fuzz citation - Google's continuous fuzzing service with manually-written harnesses as baseline
With the availability of large language models, harness generation has been explored from a learning perspective. Early feasibility studies evaluate prompting strategies~\cite{DBLP:conf/issta/ZhangZBLMXLSL24:HowEffectiveAreThey} and identify obstacles such as semantic drift~\cite{DBLP:conf/sigsoft/Jiang0MCZSWFWLZ24:WhenFuzzingMeetsLLMs}. Recent systems demonstrate automatic synthesis through coverage-guided prompt mutation for iterative refinement~\cite{DBLP:conf/ccs/LyuXCC24:PromptFuzz}, which mutates prompts based on coverage feedback but lacks semantic interpretation of coverage gaps. CKGFuzzer~\cite{DBLP:conf/icse/XuMZZCHLW25:CKGFuzzer} augments LLM reasoning with knowledge graphs of API relations, preprocessing entire API surfaces into static graphs before generation. Other approaches integrate LLM reasoning into static analysis pipelines~\cite{DBLP:journals/corr/abs-2505-03425:HGFuzzer} or combine LLM-based repair with solver-driven scheduling~\cite{DBLP:journals/corr/abs-2507-18289:Scheduzz}. However, most approaches either preprocess all information upfront (risking context saturation and prioritizing breadth over target-specific depth) or lack mechanisms for agents to iteratively query documentation and source code as reasoning progresses. Our work provides agents with query-based access to documentation, source code, and callgraph information through the Model Context Protocol~\cite{mcp}, enabling on-demand retrieval as reasoning needs emerge. While practical tools like OSS-Fuzz-Gen~\cite{oss-fuzz-gen} employ comparable multi-agent workflows with generic source access and fixed iteration counts, we provide specialized static analysis (callgraph construction, method-targeted coverage) and adaptive orchestration where coverage analysis agents determine iteration budgets dynamically. We extend feedback-driven refinement~\cite{DBLP:conf/ccs/LyuXCC24:PromptFuzz, DBLP:conf/uss/ZhangLZZZZXLL0H23:Rubick} by delegating termination decisions to agents that interpret coverage gaps, rather than applying fixed thresholds or heuristics.



\section{Conclusion}

We presented a multi-agent architecture that automates fuzzing harness generation for Java libraries through specialized ReAct agents and query-driven code analysis. Five agents decompose the workflow into research, synthesis, compilation repair, coverage analysis, and refinement, querying documentation and source code on demand through the Model Context Protocol. Method-targeted coverage instrumentation and agent-guided termination enable effective refinement without context saturation.

Evaluation on seven target methods from six widely deployed libraries demonstrates competitive coverage with manually written OSS-Fuzz baselines at practical costs of \$3.20 and 10 minutes per harness. Our harnesses discovered 3 previously unreported bugs in production libraries already integrated into OSS-Fuzz, validating that automated generation achieves sufficient quality for real vulnerability discovery. Future work includes extending to stateful APIs, identifying minimal method sets that maximize library coverage to reduce generation costs, and adapting the approach to synthesize property-based security harnesses that verify invariants beyond crash detection.

%\subsection{Threats to Validity}%
\label{subsec:threats}

\paragraph{Internal Validity}
LLM non-determinism poses challenges despite temperature=0 configuration, as Claude may produce varying outputs across runs. We mitigate this through convergence detection via code hashing and iteration limits, but different agent trajectories could yield different final harnesses with potentially different coverage profiles.

Callgraph depth configuration affects reachability analysis: we use depth-10 by default (depth-5 for large libraries like guava). Shallower depths may miss reachable methods, while deeper analysis increases agent context size and token costs. Our choice balances completeness against practical constraints.

Coverage sampling at 60-second intervals provides sufficient granularity for 12-hour campaigns, though higher-frequency sampling might reveal transient coverage dynamics. Branch coverage typically exhibits monotonic growth, limiting concerns about missed coverage spikes.

\paragraph{External Validity}
Our benchmark suite covers 7 methods across 5-6 libraries, representing diverse API patterns (parsers, serializers, utilities, generators). While these targets span common library categories, larger-scale evaluation across hundreds of targets would strengthen generalization claims about agent behavior and success rates.

The approach specifically targets JVM libraries with Maven dependency management and Javadoc documentation. Adaptation to other ecosystems (Python with pip, Rust with cargo, Go modules) requires porting static analysis tooling, documentation parsers, and build system integration. The core multi-agent architecture and MCP-based tool access should transfer, but ecosystem-specific implementation represents non-trivial engineering effort.

We exclusively use Claude 4.5 Sonnet as the underlying model. Different LLMs (GPT-4, Llama 3, Gemini) exhibit different code synthesis capabilities, reasoning patterns, and cost profiles. Future work should validate that our agent architecture and prompt design generalize across model families.

Computational constraints limited us to single 12-hour fuzzing runs per configuration. Multiple runs with different random seeds and statistical testing (e.g., Mann-Whitney U test) would provide stronger confidence in coverage comparisons and better characterize variance in fuzzing outcomes.

\paragraph{Construct Validity}
Branch coverage serves as a proxy for harness quality but doesn't guarantee bug-finding capability. A harness achieving 90\% coverage may still miss critical edge cases triggering vulnerabilities, while a 60\% coverage harness might trigger specific bugs through focused input patterns. We use coverage as a standardized metric enabling comparison across approaches, acknowledging its limitations.

Method-targeted coverage instrumentation improves alignment with target-specific fuzzing goals but may undercount relevant coverage in utility methods or deeply nested call chains invoked by the target. This trade-off prioritizes focused metrics over comprehensive library coverage.

Cost measurements include only LLM API charges (tokens × price per token) but exclude infrastructure costs: callgraph construction, documentation parsing, and build environment preparation. Total deployment cost in production would be higher, though these fixed costs amortize across multiple targets within the same library ecosystem.


%%
%% The next two lines define the bibliography style to be used, and
%% the bibliography file.
\bibliographystyle{ACM-Reference-Format}
\bibliography{main}


\end{document}
\endinput
%%
%% End of file `sample-sigconf.tex'.